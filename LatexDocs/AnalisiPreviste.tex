\documentclass{article}

\usepackage[utf8]{inputenc}
\usepackage[italian]{babel}
\usepackage[hyphens]{url}

\usepackage{graphicx}

\usepackage{listings}

\usepackage{biblatex}
\addbibresource{References.bib}


\usepackage{paralist}
\usepackage{layout}

\usepackage{graphicx}
\usepackage{fancyhdr}

\usepackage[a4paper,top=1cm,bottom=1cm,outer=3cm,inner=1cm,verbose,headheight=1cm,heightrounded]{geometry}
\setlength{\marginparwidth}{2.5cm} %per farci stare todonotes

\usepackage{refcheck}
\usepackage[colorinlistoftodos]{todonotes}
\usepackage[breaklinks=true]{hyperref} % interessante https://en.wikibooks.org/wiki/LaTeX/Labels_and_Cross-referencing

\title{Specifica analisi previste per l'attività di tesi}
\author{Marco Belotti}

\begin{document}
\maketitle
\tableofcontents
\listoftodos
\newpage

In questo documento andrò ad elencare e descrivere le varie analisi che intendo affrontare nel corso della mia tesi. 


\section{Introduzione}
\label{sec:introduzione}
Nel corso di questo studio intendo affrontare delle analisi sui livelli degli inquinanti e su come questi vengono influenzati da diversi fattori, facendo una ricerca su quali sono le condizioni e le misure che più influenzano i valori regisgtrati. È ormai verificato che dal 1990 in poi i livelli di quasi tutti gli inquinanti abbiano subito un forte calo (SOx -94\%; NOx -66\%; CO -68\%)\cite{iir2020}, sostanzialmente dovuto a due fattori principali: l'introduzione di nuove normative (vedi sottosezione relativa\ref{subsec:normative})
%\todo{atrent: ref o nota sul progresso delle categorie EuroX ecc.}
atte a limitare le emissioni prodotte dalle attività umane e l'innovazione tecnologica che ha coinvolto qualsiasi settore (trasporti, industrie, produzione energetica, ecc..). Ritengo quindi che in uno studio che vuole indagare sull'andamento degli inquinanti nel corso degli anni questi due siano i fattori che devono avere maggiore considerazione.

Principalmente l'idea è quella di occuparsi di tre temi: inquinanti dovuti al traffico e misure adottate, produzione energetica e fonti rinnovabili e come si è evoluta la situazione durante l'epidemia di COVID-19. Andrò quindi ad analizzare l'impatto delle misure messe in atto e dell'innovazione tecnologica riguardanti questi tre apsetti, cercando di verificare quali reali effetti si sono avuti sui livelli degli inquinanti registrati.

Credo comunque che in prima fase sia necessario delineare un quadro della situazione, presentando dei grafici che mostrino effettivamente i trend in calo delle concentrazioni, andando ad analizzare tutto lo storico dei dati a disposizione. L'idea sarebbe quella di controllare ogni inquinante e produrre un grafico riportante l'andamento anno per anno dei valori registrati, rappresentadoli attraverso dei box plot, in modo che sia più visibile il range dei valori che semplicemente la media aritmetica degli stessi. È infatti importante ricordare che i livelli dei vari inquinanti sono soggetti a molte variazioni, dovute alle condizioni climatiche che possono favorirne l'accumulo o la dispersione e a molti altri fattori, ed è quindi opportuno considerare degli intervalli pittuosto che un unico valore di riferimento come potrebbe essere la media aritmetica.
Inoltre potrebbe essere utile andare ad analizzare la frequenza dei superamenti delle soglie che hanno fatto registrare i diversi inquinanti, in modo da capire quali sono i più critici e che sicuramente meritano analisi più approfondite. Ad esempio gli ossidi di zolfo (SOx), che una volta erano tra gli inquinanti più preoccupanti, ormai sono stati portati a livelli trascurabili e quindi potrebbe risultare inutile basare su di essi un'analisi relativa ad un periodo recente.

Si potrebbe inoltre cercare di mettere in correlazione (ad esempio mediante la presentazione di scatter plot, così come usando il calcolo dell'indice di correlazione) i livelli degli inquinanti con i rilevamenti effettuati dalle stazioni meteo. Questo dovrebbe permetterci di constatare come determinate condizioni climatiche influenzino i livelli degli inquinanti e può sicuramente aiutarci nella comprensione di quali siano i fattori puù influenti. Questa parte potrebbe comunque essere integrata nel processo di normalizzazione dati, descritto più avanti (vedi sezione 'Aspetti importanti').

%\todo{atrent: da qualche parte fare anche un'analisi degli abbassamenti (normativi) delle soglie nel corso del tempo, che hanno fatto sembrare l'inquinamento in salita, riportando bib che spieghi perché lo hanno fatto (secondo me inutilmente, ma non posso dimostrarlo)}
Credo che questo aspetto si possa integrare nella sopracitata analisi dei superamenti delle soglie. Si potrebbe controllare sia il numero dei superamenti che il rapporto tra la media dei valori registrati e il valore della soglia. Questo ci permetterebbe di vedere come le concentrazioni di alcuni inquinanti, che negli ultimi anni potrebbero aver fatto registrare un numero maggiore di superamenti (e quindi anche un rapporto più alto), in realtà siano comunque in calo, anche se sembrerebbe il contrario. Gli abbassamenti delle soglie nel tempo sono messi in atto in accordo alla politica europea di abbattimento delle emissioni di inquinanti (e gas serra), che ormai sono in atto da più di un decennio. L'obbiettivo dichiarato di queste politiche, che prevedono anche piani a lungo termine (addirittura fino al 2050), è quello di puntare sempre a una migliore qualità dell'aria che comporti sempre meno rischi per la salute umana.


\subsection{Normative}
\label{subsec:normative}
In questa sezione indicherò brevemente le normative di riferimento nel corso dei miei studi.

Per quanto riguarda le normative italiane e regionali, nella tesi\cite{scolari2017evoluzione} è sicuramente contenuto un buon riassunto di tutte le normative introdotte nel corso del tempo. Sostanzialmente, comunque, si è sempre continuato ad aumentare il numero degli inquianti considerati, abbassando progressivamente le soglie con l'obbiettivo di raggiungere standard elevati (forse anche troppo?), oltre che a limitare l'utilizzo di particolari combustibili e all'obbligo del controllo delle emissioni. 

Per quanto riguarda il traffico è sicuramente importante considerare e conoscere gli standard europei sulle emissioni inquinanti, ovvero le famose categorie EuroX che vengono applicate ai veicoli per indicare il loro livello emissivo. Attualmente la categoria in vigore è quella degli Euro6, introdotta nel 2011 e di cui tuttora non è previsto il superamento, che con le progressive revisioni è arrivata a raggiungere dei livelli emissivi davvero bassi, che rendono quindi le automobili ancora meno responsabili dell'inquinamento registrato. 

Per quanto riguarda il settore energetico nel corso degli anni novanta sono state introdotte una serie di misure atte a limitare l'inquinamento prodotto da questo settore (ad esempio il DPR203/88), con particolare riferimento a SOx, NOx e PM10. Nel corso degli anni 2000 questi limiti sono stati poi continuamente abbassati, è stato incentivato il passaggio all'utilizzo del metano come alternativa al gasolio ed in generale si è vista l'adozione di una serie di tecniche atte ad abbattere le emissioni prodotte dalle centrali. 
Per regolare l'utilizzo di legna e pellet le prime limitazioni sono state imposte col Decreto Regionale DGR 3024/2006 e poi continuamente aggiornate fino ad arrivare all'approvazione dell'Accordo del Bacino Padano, che ha imposto limitazioni sull'uso e sull'efficienza degli impianti di riscaldamento funzionanti a biomasse.

Nel corso delle analisi legate all'epidemia COVID i procedimenti da considerare saranno tutti i decreti che progressivamente hanno bloccato le varie attività, fino ad arrivare a quello finale del 22 Marzo, in cui è stato disposto il blocco totale di qualsiasi attività ritenuta non essenziale.


\section{Effetti del traffico}
\label{sec:traffico}
Un'analisi che è sicuramente interessante affrontare è quella riguardante l'efficacia dei blocchi del traffico ed il loro impatto sui livelli degli inquinanti. 

Gli inquinanti maggiormente collegati al traffico sono gli ossidi di azoto (NOx), il monossido di carbonio (CO) e le polveri sottili (PM10 e PM2.5), quindi saranno da considerare stazioni e registrazioni riguardanti questi elementi.

Per capirne l'andamento è innanzitutto fondamentale verificare la presenza di un trend nel corso degli anni, in modo da poter stimare quali sarebbero i valori registrati in mancanza di interventi atti a limitarli. Una delle possibili tecniche utilizzabili è l'applicazione del test di Kendall corretto per la stagionalità, come suggerito dallo studio di ISPRA \cite{cattani2014analisi}, che permette di verificare l'esistenza di un trend eliminando la relazione che i livelli degli inquinanti hanno con i mesi dell'anno, in quanto la stagione influisce fortemente sulle concentrazioni registrate per i vari componenti.   

Per verificare l'utilità di questo tipo di misure prevedo l'utilizzo di un doppio confronto: il primo andrà a verificare, su una stazione che è stata coinvolta dal blocco del traffico, se tra prima e dopo l'applicazione della restrizione si possono notare degli effetti oppure se la situazione rimane invariata.
%\todo{atrent: tipo "gruppo di controllo" nel tempo e nello spazio, corretto?}
Più che considerare un'unica stazione, forse è il caso di prendere un insieme di stazioni che siano state colpite dalla misura, per verificare come si sono comportati i livelli registrati nel corso del tempo, in modo da poter identificare meglio eventuali effetti derivati dal blocco. Se, per esempio, successivamente all'applicazione della misura tutte le stazioni mostrano dei trend in calo allora si può pensare che la stessa abbia avuto effetto, mentre se su qualche stazione si nota un calo ma su altre c'è un aumento è probabile che la restrizione non abbia avuto efficacia.

Il secondo confronto da mettere in campo va a comparare i dati di una stazione coinvolta dal blocco con quelli di una che non è stata interessata, valutando se nello stesso periodo ci sono differenze di comportamento dei livelli degli inquinanti che possano suggerire un'effettiva efficacia delle misure.
%\todo{atrent: o paesi lontani, es. milano e bormio}
Per la scelta della seconda stazione si hanno due possibilità: usare una stazione simile a quella coinvolta dal blocco, come potrebbe essere un'area più periferica della città che non è stata coinvolta dalla misura, oppure scegliere una stazione in una località lontana e con caratteristiche piuttosto differenti, come quelle di località montane o rurali. Sono tutte e due valide scelte, la prima ci permette di verificare se una misura presa può aver portato a dei reali miglioramenti oppure se applicare delle misure solo a determinate zona di una città sia inutile, mentre la seconda ci permette di fare un confronto con una situazione completamente diversa in cui è possibile verificare quale è stata l'evoluzione delle concentrazioni senza che ci siano stati interventi. Credo siano entrambe valide scelte, quindi la cosa migliore è sicuramente utilizzare entrambi i confronti nel corso delle nostre analisi.

Saranno sicuramente da considerare anche le diverse tipologie di blocchi che vengono imposti, che sostanzialmente si possono raggruppare in due categorie: misure a lungo termine (Area C, Area B, Ztl, ecc..) e misure temporanee (blocchi del traffico di qualche giorno), per verificare se e in quali casi queste misure portano a dei benefici reali. 

L'evoluzione tecnologica nel mondo dell'auto ha comportato una serie di miglioramenti che hanno drasticamente fatto calare le emissioni prodotte dai veicoli, sia a livello di componenti che nei carburanti utilizzati. Dal 1990, infatti, inquinanti come il CO, di cui una volta il settore dei trasporti era responsabile del 65\% delle emissioni totali, hanno visto un netto calo, grazie all'applicazione di innovazioni come le marmitte catalitiche. Un altro esempio è quello degli SOx, derivati dalla combustione dello zolfo, che una volta era presente in alte concentrazioni nei carburanti, mentre ora il suo impiego è limitato per legge. Si può quindi fare un'analisi sui livelli registrati da questi due inquinanti, per vedere come l'applicazione di queste norme e l'ammodernamento della flotta di veicoli abbiano portato a un abbassamento delle concentrazioni misurate nel corso degli anni.

\subsection{Inquinanti da considerare}
Come indicato dall'inventario pubblicato da ISPRA \cite{iir2020}, gli inquinanti più rilevanti per quanto riguarda il traffico sono (dati relativi al 2018): 
\begin{itemize}
	\item NOx, con il 43\% della responsabilità sul totale delle emissioni
	\item CO, con circa il 20\% della responsabilità sul totale delle emissioni
	\item PM10, con il 12\% della responsabilità sul totale delle emissioni
	\item PM2.5, con il 10\% della responsabilità sul totale delle emissioni
\end{itemize}
Questi dati sono molto importanti, in quanto ci permettono di capire di quali inquinanti conviene occuparsi durante le nostre analisi, soprattutto ci indicano quali bisogna monitorare per verificare il funzionamento delle norme relative al traffico. 

Inoltre nello stesso rapporto sono riportati una serie di informazioni molto interessanti per quanto riguarda il traffico e la tipologia di veicoli che circolano sulle strade. Si potrà quindi pensare ad una loro integrazione nel corso delle nostre analisi.


\section{Produzione energetica e fonti rinnovabili}
\label{sec:energia}
Cercare di analizzare l'impatto dell'utilizzo di fonti d'energia rinnovabili sulla qualità dell'aria non è sicuramente un'attività semplice, soprattutto perché la produzione di energia pubblica influenza sui livelli degli inquinanti in modo ridotto (vedi sezione successiva).

Intendo comunque provare a fare un'analisi legata ai livelli degli inquinanti più caratteristici di questo settore, per cercare di evidenziare delle relazioni tra i trend in calo e il sempre maggior impiego di fonti rinnovabili, soprattutto per quanto riguarda quegli inquinanti che una volta erano largamente prodotti da questo settore ed invece ora, grazie all'innovazione tecnologica, alle nuove norme e all'impiego di energie pulite, sono ampiamente calati, come ad esempio SOx, NOx e PM10.
La difficoltà di questo tipo di analisi è legata al fatto che comunque tutti gli inquinanti seguono dei trend in calo, perciò riuscire a calcolare quanto può essere stata l'influenza del maggior impiego di fonti rinnovabili non sarà sicuramente semplice. Un'aiuto può venire dai dati che Terna, la società che gestisce le reti ad alta tensione in Italia, fornisce relativi ai consumi energetici nazionali, indicando anche la ripartizione sulle varie fonti di generazione. Si potrebbe quindi provare a verificare se nel corso degli anni la quantità di energia prodotta da centrali termiche si è ridotta, per poi successivamente andare a controllare se questo calo può essere compatibile con i trend registrati dagli inquinanti, in particolar modo di quelli legati a questo tipo di produzione. In alternativa, o come analisi di supporto, è possibile andare a monitorare gli andamenti degli inquinanti, ovviamente considerando quelli caratteristici di questo settore, in relazione con le varie normative che nel corso degli anni sono state adottate per il loro contenimento. Un esempio potrebbero essere quelle norme che sono intervenute sulle concentrazioni di zolfo contenute nei combustibili, soprattutto quelli che una volta erano largamente usati per il riscaldamento domestico e per la produzione enrgetica, per controllare se dall'applicazione di questi provvedimenti si possono notare dei miglioramenti concreti sui livelli registrati. 

Un'altra analisi che si può condurre per cercare di dimostrare l'impatto dell'utilizzo di determinate fonti sulla qualità dell'aria e l'importanza delle norme adottate, è quella legata all'utilizzo di stufe e caminetti a legna come forme di riscaldamento domestico. ARPA Lombardia, nell'inventario regionale sulle emissioni degli inquinanti atmosferici\cite{inemar2017}, indica la legna come la responsabile del 50\% delle emissioni di PM10 e del 25\% delle emissioni di CO. Si potrà quindi andare a verificare le rilevazioni effettuate per questi inquinanti in località dove questa forma di riscaldamento è maggiormente utilizzata (quindi località montane e rurali), per confrontarle con quelle di una località in cui l'utilizzo della legna come forma di riscaldamento domestico viene limitato con obblighi sull'efficienza dell'impianto e sulle emissioni prodotte (aree urbane e paesi sotto i 300 metri di quota), concentrandosi sulle rilevazioni fatte in periodo invernale. Dovremo sicuramente tener conto che tra le due località le concentrazioni normalmente possano già differire, quindi al posto che confrontare semplicemente i valori registrati si può pensare di calcolare la variazione delle concentrazioni tra la stagione estiva e quella invernale, che dovrebbe essere più marcata nel caso in cui in inverno si utilizzi maggiormente la legna. Inoltre si potrà andare a verificare su base diurna come sono distribuite le concentrazioni e in quali momenti si hanno quelle maggiori. Nella località dove l'utilizzo di legna non è permesso dovrebbero corrispondere ai momenti di maggior traffico, mentre nell'altra potrebbero essere più alte nei momenti di maggior utilizzo dei riscaldamenti domestici (quindi sera e notte).

\subsection{Inquinanti da considerare}
Come indicato dall'inventario pubblicato da ISPRA \cite{iir2020}, gli inquinanti più rilevanti per quanto riguarda la produzione di energia sono (dati relativi al 2018): 
\begin{itemize}
	\item SOx, con l'8\% della responsabilità sul totale delle emissioni (ma nel 1990 questa percentuale era di oltre il 50\%)
	\item NOx, con circa il 4\% della responsabilità sul totale delle emissioni
	\item SOx, NOx e PM10 prodotti da questo settore negli ultimi anni hanno comunque visto un forte calo
\end{itemize}
Come già accennato l'incidenza di questo settore sul totale delle emissioni è ormai contenuta, ma non è sempre stato così storicamente. Gli inquinanti che sicuramente sarà più interessante ed utile analizzare per capire l'importanza delle norme e delle tecnologie applicate nel corso degli anni sono sicuramente gli SOx.


\section{Epidemia CoVID-19}
\label{sec:covid}
Partendo da quanto già approfondito da ARPA Lombardia nel suo studio \cite{arpaCovid}, sarà interessante andare ad analizzare i dati sulla qualità dell'aria durante l'epidemia di COVID-19, soprattutto una volta che avremo a disposizione il quadro completo delle rilevazioni. Bisognerà effettuare dei confronti con gli andamenti degli anni precedenti, con particolare attenzione ad analizzare gli stessi mesi dell'anno, per capire se e quanto le progressive limitazioni imposte dal Governo possano aver influito sulla qualità dell'aria.

Ovviamente nel corso di questa analisi, come suggerito dal dottor Carlo Bozzetti, bisognerà considerare tutti i vari fattori che possono aver influenzato sui valori misurati, in modo da avere una stima più precisa. Sicuramente sarà da considerare il trend in discesa che tutti gli inquinanti registrano, così come saranno da considerare particolari condizioni atmosferiche che possono aver portato a determinati livelli registrati. È quindi pensabile che anche per questa fase bisognerà cercare di fare un lavoro di normalizzazione e pulizia dei dati.

Un'altra analisi che si può fare, legata al cambiamento di abitudini della popolazione dovuto alle limitazioni imposte, è vedere come su base giornaliera possono essere cambiati i livelli di inquinamento. Tipicamente, infatti, le fasce orarie caratterizzare da maggiori livelli di traffico sono chiaramente quelle in cui si registrano le concentrazioni maggiori. In questo periodo, invece, in cui i livelli di traffico sono completamente ridotti, sarà interessante vedere se c'è stato qualche cambiamento ed eventualmente si potrà anche provare ad indagare sui risultati, per ricercare quali possono essere le cause dei valori registrati (riscaldamento, maggior consumo energetico, ecc..).

Proprio per questo calo del traffico sarebbe utile andare ad analizzare i dati di una stazione collocata vicino ad un'importante arteria stradale, confrontandoli magari con una stazione urbana, per vedere se si possono riscontrare differenze significative o se invece i valori registrati da entrambe tendono a seguire uno stesso trend.

%\todo{atrent: si potrà tentare di stimare il famoso "fondo naturale"? forse difficile perché in realtà non tutte le attività si sono fermate}
Sarebbe sicuramente interessante provare a stimare il "fondo naturale" e credo che questo potrebbe essere il periodo migliore per farlo, visto il forte calo della attività umane. Purtroppo però è anche vero che non tutte le attività si sono fermate, così come gli inquinanti derivati dai riscaldamenti domestici e dalla produzione energetica non dovrebbero essere calati così drasticamente. Si potrebbe comunque pensare di indagare su alcuni inquinanti specifici derivanti principalmente da traffico e produzione industriale, che potrebbero essere quelli maggiormente colpiti da questo periodo di lockdown, per vedere se si può condurre qualche analisi in merito a questo aspetto. Sicuramente quando si avrà a disposizione tutto il quadro della situazione sarà più facile capire su quali inquinanti può valer la pena di indagare, che saranno quelli che avranno fatto registrare i maggiori cali.


\section{Prime idee per quanto riguarda la parte di codice}
Come già previsto intendo utilizzare Python e la libreria matplotlib per tutta la parte di raccolta, elaborazione e presentazione dati. I dati forniti da ARPA sono liberamente accessibili anche attraverso l'utlizzo di API, che permettono di fare direttamente delle query sul dataset per farsi ritornare i dati desiderati. Questo potrebbe essere molto d'aiuto, in quanto dovrebbe evitarci la necessità di produrre un file CSV per contenere i dati di ogni sensore, alleggerendo così il carico elaborativo dell'applicazione.


\section{Aspetti importanti}
Di seguito una serie di considerazioni riguardanti tutte le analisi previste.
\subsection{Normalizzazione dei dati}
Come già evidenziato da diversi studi, ad esempio \cite{hoogerbrugge2010trends} e \cite{grange2019using}, la normalizzazione dei dati sull'inquinamento relativamente alle condizioni metereologiche è sicuramente una tencica molto importante, che ci permette di indagare con più precisione sia sui trend fatti registrare dalle concentrazioni che sull'efficacia di misure prese, poichè ci consente di eliminare la variabilità causata da tali elementi. Come è ben noto, infatti, pioggia, vento, umidità, pressione atmosferica e temperatura influiscono tutti sui livelli di tutti gli inquinanti, arrivando a stravolgerne completamente i valori, quindi sarebbe sicuramente un vantaggio poter tener conto di questa loro influenza nell'analisi dei valori registrati.

Il metodo che vorrei usare per effettuare questa normalizzazione sarebbe l'applicazione di una regressione lineare multipla, che è una tecnica che ci permette di stimare il valore di una variabile basandoci sui valori di un numero di variabili indipendenti. Con la sua applicazione saremmo quindi in grado di normalizzare i dati in base alle condizioni meteo registrate, così da poter poi condurre analisi più precise. Penso che questa sia una tecnica valida, che è già stata largamente usata in molti campi, soprattutto quello delle intelligenze artificiali, e che non presenta un'estrema complessità. Proprio per questa sua "notorietà" sono già presenti dei package di Python che permettono di eseguire questo tipo di analisi in modo abbastanza semplice, quindi il suo utilizzo non richiederebbe nemmeno troppo sforzo implementativo. Credo sia quindi un'ottima scelta, che mi permetterebbe anche di sperimentare con tecniche riguardanti le intelligenze artificiali con cui sarei molto curioso di lavorare. Un'applicazione di un metodo di questo tipo è stata usata nello studio olandese sorpacitato\cite{hoogerbrugge2010trends}, che sicuramente può essere considerato una buona base di partenza.

Credo che questo della normalizzazione sia un passaggio molto importante, da affrontare come attività iniziale, in modo da avere poi un modello per la normalizzazione dei valori rilevati che ci permetta di affrontare qualsiasi tipo di analisi con più precisione. Sarebbe richiesta una fase dedicata, necessaria per approfondire la tecnica e per metterla in pratica, ma penso che i vantaggi che se ne otterrebbero sarebbero davvero importanti per tutto il lavoro che intendo affrontare poi nel corso del mio studio.

Per quanto riguarda lo studio dei trend nelle concentrazione registrate, spunti importanti sono dati dai rapporti di ISPRA\cite{cattani2018analisi}\cite{cattani2014analisi}, in cui si è ricercata una tecnica di analisi dei dati che permetta di fare delle analisi di tipo quantitativo sugli andamenti degli inquinanti. Come già accennato la tecnica applicata è il test di Kendall corretto per la stagionalità, il cui utilizzo può essere valido anche per il mio studio quando ci sarà la necessità di quantificare il calo registrato dalle concentrazioni. Stavo pensando ad una sua possibile applicazione nella verifica di efficacia dei blocchi del traffico, ma dovrei pensarci meglio ed approfondire alcuni dettagli per verificarne la fattibilità.

\subsection{Difficoltà nell'analisi dei risultati}
Nelle analisi previste sicuramente tra gli obbiettivi c'è quello di cercare delle relazioni tra misure adottate (blocchi del traffico, maggior utilizzo di energie rinnovabili) ed i livelli di inquinamento. Questo non sarà sicuramente un compito semplice, soprattutto per quanto riguarda quelle analisi che riguardano periodi brevi e quindi molto più soggetti alla variabilità dei livelli degli inquinanti. Il nostro lavoro dovrà quindi tenere conto di questo aspetto, cercando di allargare il più possibile l'insieme dei dati di confronto, in modo da "attutire" questa variabilità. Ad esempio, per quanto riguarda l'epidemia COVID, il confronto dei livelli registrati nel mese di Marzo dovrà essere almeno con i dati dello stesso mese degli ultimi anni e non sono con quelli dell'anno scorso, così da avere un miglior quadro della situazione.

Un altro aspetto che sarà importante ricordare in queste analisi è che tutti i livelli presentano dei trend e quindi bisognerà tenerne conto per avere una stima più precisa dei reali ribassamenti/innalzamenti registrati.

Inoltre, quando si trattano analisi che cercano di stimare l'influenza di determinate misure sulla qualità dell'aria, non sarà semplice fornire dei risultati certi e bisognerà essere cauti a non trarre conclusioni errate o influenzate da altri fattori. Sarà quindi opportuno che ogni attività prevista veda lo svolgimento di più analisi, che possano darci una certa confidenza sulla validità dei risultati ottenuti, senza cadere nel rischio di presupporre correlazioni basate solo su un semplice confronto.


\section{Sommario attività}
Qui di seguito andrò ad elencare quindi le varie attività che prevedo di affrontare nel corso del mio studio.

Il primo passaggio da affrontare è sicuramente quello relativo alla normalizzazione dei dati. Sarà da mettere bene a punto una tecnica precisa, che come già accennato sarà basata sulla regressione lineare, in modo da avere poi un modello di normalizzazione valido e applicabile in tutte le altre analisi. Siccome questa tecnica richiede l'analisi delle registrazioni effettuate per poter stabilire quali relazioni legano le rilevazioni meteo con i livelli degli inquinanti, sarà sicuramente da affrontare anche il tema del recupero dei dati dai datasets, andando a creare delle funzioni che saranno poi usate anche nel resto del nostro studio.

Una volta che avremo un modello soddisfacente per la normalizzazione dei dati potremo passare ad affrontare le analisi descritte nella sezione 'Introduzione'\ref{sec:introduzione}. Non sono analisi complicate e non credo che richiederanno troppo sforzo, ma ci serviranno per avere un quadro generale della situazione prima di affrontare gli altri temi. Inoltre sarà possibile vedere come l'applicazione della normalizzazione andrà a variare i valori registrati, permettendoci di verificare come cambia il quadro della situazione dopo l'applicazione di questa tecnica.

Successivamente intendo affrontare le analisi descritte nella sezione relativa al traffico\ref{sec:traffico}. I confronti da affrontare saranno quelli specificati, eventualmente da supportare con ulteriori studi se li riterremo necessari. In questa fase sarà necessario implementare diverse funzionalità per il recupero e l'elaborazione dei dati, che, se ben sviluppate, saranno poi probabilmente utilizzabili anche per tutto il resto dell studio, andando solo a modificare le parti necessarie.

Andrà poi affrontato il settore della produzione energetica, andando a svolgere le analisi previste nella relativa sezione\ref{sec:energia}.

Come ultima analisi, sperando di avere un quadro completo della situazione, si potrà andare ad analizzare l'evoluzione degli inquinanti nel periodo dell'epidemia di COVID-19. Anche per questa fase le analisi previste sono quelle già descritte nella sezione relativa\ref{sec:covid}, ma potrebbero essere espanse o modificate in base all'evoluzione della situazione. ARPA Lombardia, dopo aver pubblicato la prima analisi\cite{arpaCovid}, aveva promesso di riaffrontare la questione una volta che ci sarebbe stato a disposizione il quadro completo dell'evoluzione, quindi se dovessero uscire nuovi studi saranno sicuramente da considerare e valutare se eventualmente sarà necessario apportare qualche modifica agli studi previsti. 
 

%\bibliographystyle{ieeetr}
%\bibliography{References}
\label{bib-begin}
\printbibliography
\end{document}
