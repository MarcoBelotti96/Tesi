% scopiazzato dal template di Matteo Longeri (grazie!)
%%%%%%%%%%%%%%%%%%%%%%%%%%%%%%%%%%%%%%%%%%%%%%%%%%%%%%
\documentclass[a4paper]{report}
% o article, book, ...



%%%%%%%%%%%%%%%%%%%%%%%%%%%%%%%%%%%%%%%%%%%%%%%%%%%%%%
% packages...
\usepackage[utf8]{inputenc}
\usepackage[english,italian]{babel}
\usepackage[hyphens]{url}

% Per generare il file PDF aderente alle specifiche PDF/A-1b. Verificarne poi la validità.
%\usepackage[a-1b]{pdfx}

\usepackage{hyperref}
\usepackage{graphicx}


% Per inserire testo a caso in attesa di realizzare i capitoli
\usepackage{lipsum}


%%%%%%%%%%%%%%%%%%%%%%%%%%%%%%%%%%%%%%%%%%%%%%%%%%%%%
\begin{document}

% Frontespizio
\begin{titlepage}
\begin{center}
\includegraphics[width=\textwidth]{Logo.jpg}\\
{\large{\bf Corso di Laurea Triennale in Informatica}}
\end{center}
\vspace{12mm}
\begin{center}
{\huge{\bf TITOLO}}\\
\vspace{4mm}
{\huge{\bf TITOLO}}\\
\vspace{4mm}
{\huge{\bf TITOLO}}\\
\end{center}
\vspace{12mm}
\begin{flushright}
{\large{\bf Tesi di Laurea di:}}\\
{\large{\bf Marco Belotti}}\\
{\large{\bf Matr. 871440}}\\
\end{flushright}
\vspace{4mm}
\begin{flushleft}
{\large{\bf Relatore:}}\\
{\large{\bf Andrea Trentini}}\\
\vspace{4mm}
%{\large{\bf Correlatore:}}\\
%{\large{\bf CORREL}}\\
\end{flushleft}
\vspace{12mm}
\begin{center}
{\large{\bf Anno Accademico 2019/2020}}
\end{center}
\end{titlepage}


\tableofcontents


\chapter{Introduzione}

L'inquinamento atmosferico è un tema che viene spesso discusso e dibattuto, visto che respirare un'aria pulita è considerato un requisito basilare per la buona salute umana. Da anni il WHO\cite{world2006air} si occupa di studiare i potenziali effetti dell'inquinamento sulla salute umana, cercando di stimare che tipo di impatto possa avere un'esposizione prolungata ad alte concentrazioni.  
Chiaramente non è facile riuscire a stabilire con certezza il collegamento tra causa (esposizione agli inquinanti) ed effetto (malattie e morte) poichè spesso i danni si manifestano solo a seguito di ripetute e prolungate esposizioni, così come non è facile riuscire a stabilire con certezza che una morte sia stata causata esplicitamente dall'inquinamento atmosferico.  
L'inquinamento atmosferico è un fenomeno molto complicato da analizzare, poichè sia per quanto riguarda la produzione e l'emissione in atmosfera che per la dispersione entrano in gioco molti fenomeni e fattori, che hanno scale spaziali molto differenti e che è difficile riuscire ad analizzare correttamente.  
Se in alcune zone del mondo, soprattutto quelle dei paesi in forte via di sviluppo come il sud-est asiatico o l'India, l'inquinamento rappresenta ancora un problema molto grave causato dall'aumento del fabbisogno energetico che è stato soddisfatto incrementando l'uso di combusitibili fossili, in Europa (come in altre parti del mondo) negli ultimi 25 anni si è registrato un disaccoppiamento tra la crescita economica e le emissioni dei principali inquinanti, causato da un maggiore impegno nel loro contenimento da parte dei governi, che hanno imposto limitazioni e misure sempre più stringenti.\cite{cattani2014analisi}  
È quindi normale che sia la comunità scientifica che le autorità incaricate di decidere le misure di contenimento da mettere in campo siano interessate a capire quale sia il reale andamento delle concentrazioni di inquinanti, cercando di identificare quali possano essere i metodi più efficaci e quanto le variazioni registrate siano attribuibili specificatamemente ai cambiamenti delle emissioni umane rispetto agli effetti di altri fattori come il clima che possono avere un'influenza molto maggiore sulle concentrazioni registrate.\cite{porter2001ozone}  
Quasi tutti gli inquinanti di maggior interesse, come gli ossidi di azoto, il biossido di zolfo, il monossido di carbonio, il particolato e l'ozono sono infatti caratterizzati da una forte variabilità stagionale che, per esempio, non ci permette di confrontare valori registrati durante la stagione estiva con quelli invernali, così come l'azione di agenti atmosferici come il vento e la pioggia possono portare ad abbattimenti delle concentrazioni. Nella pianura padana, per esempio, è infatti frequente assistere, specialmente nei mesi invernali che sono già quelli più critici per le concentrazioni di molti inquinanti, a lunghi periodi senza precipitazioni che portano a grossi aumenti delle concentrazioni, che però vengono rapidamente abbattute all'arrivo della pioggia. %RIVEDI  
Per poter quindi trarre delle conclusioni oggettive sullo stato della qualità dell'aria, riuscendo ad identificare le cause delle variazioni che si registrano e quindi a capire anche l'efficacia delle misure prese, nel corso degli ultimi anni molti studi si sono avvalsi di diverse tecniche statistiche/probabilistiche per poter fare un'analisi oggettiva dell'inquinamento atmosferico.\cite{cattani2014analisi}\cite{rao1994detecting}\cite{libiseller2005meteorological}\cite{pbl2010trends}
Riuscire ad avere a disposizione tecniche e modelli che ci permettono di analizzare l'andamento delle concentrazioni e ad identificare le cause che possono cambiare questo andamento ci permette di capire quali siano le aree più critiche su cui andare ad intervenire ma ci può anche permettere di identificare gli interventi più efficaci e sui quali conviene investire maggiormente. La crescita economica e l'aumento del fabbisogno energetico rendono quindi fondamentale il controllo delle emissioni, per poter mantenere la qualità dell'aria a livelli non pericolosi per la salute umana e al tempo stesso non bloccare le attività economiche.  
Qualsiasi studio sui trend degli inquinanti, così come qualsiasi conclusione che ne può derivare, non possono quindi non tener conto degli effetti della meteorologia sulle concentrazioni, poichè i risultati che si otterrebbero correrebbero il rischio di non rappresentare il cambiamento nelle emissioni antropiche ma anche una variazione nei fenomeni atmosferici e negli agenti climatici.  
Il nostro obbiettivo è quindi stato quello di ricercare una tecnica per normalizzare i dati sull'inquinamento rispetto alla meteorologia e al clima, ovvero che fosse in grado di eliminare l'influenza degli agenti atmosferici come pioggia, vento e radiazione solare e della stagionalità che porta ad avere condizioni più o meno favorevoli alla dispresione dalle concentrazioni registrate degli inquinanti, in modo da arrivare poi ad avere una serie storica su cui poter fare analisi sull'efficacia delle misure prese nel corso degli anni per il contenimento delle concentrazioni.  
Un metodo comune per riuscire a fare questa "pulizia" dei dati si basa sulla creazione di modelli statistici che tramite l'uso di diverse variabili predittrici, come le misure meteorologiche su vento, precipitazioni e radiazione solare e variabili temporali legate al giorno dell'anno e alla date, cercando di prevedere quale potrebbe essere la concentrazione registrata di un inquinante. Se i modelli che riusciamo a creare si rivelano abbastanza precisi nell'attività predittoria e in grado di spiegare buona parte della varianza delle concentrazioni allora possiamo usarli con una certa confidenza per eliminare gli effetti della meteorologia dalle concentrazioni di inquinanti.  
Questo procedimento viene però complicato dal fatto che le dinamiche che portano la meteorologia a modificare le concentrazioni varino molto a seconda della località e che quindi differenti contesti, come ad esempio possono essere una città rispetto ad una località montana o situata a fondo valle, necessitino trattazione specifiche a seconda del caso. Inoltre è chiaro come le variabili meteorologiche tra di loro siano collegate e correlate e questo porta all'insorgere di una serie di problemi come la normalità, la multi collinearità e l'indipendenza che rendono la trattazione con diversi metodi molto difficile e complicata.  
Nel corso degli ultimi tre decenni nel mondo dell'informatica ha preso campo un'area detta machine learning (ML), che fa uso dei computer e della statistica per creare dei modelli predittori che potessero porsi come alternativa ai classici metodi statistici già esistenti. Sono quindi state sviluppate tecniche di ML non parametriche che, grazie all'uso dei computer per analizzare grosse moli di dati, permettono di arrivare alla creazione di modelli con buone capacità predittive anche senza occuparsi dei problemi che invece caratterizzano gli approcci puramente statistici.  
ARPA Lombardia possiede, sul territorio regionale, una rete di 85 stazioni fisse che, per mezzo di analizatori automatici, sono in grad di fornire rilevazioni ad intervalli di tempo regolari, permettendoci di creare quindi un dettagliato catalogo dei valori delle concentrazioni nel corso del tempo. Tutte queste rilevazioni vengono poi rese accessibili liberamente tramite archivi online, per permetterne la libera consultazione da parte di tutti i cittadini.  
Nel nostro studio abbiamo quindi fatto uso di questi dati, unitamente a quelli delle variabili meteorologiche sempre resi disponibili da ARPA Lombardia, per creare tramite l'uso di una tecnica di machine learning chiamata random forest dei modelli predittivi che ci permettano di arrivare a fare questa normalizzazione delle concentrazioni di inquinanti rispetto alla meteorologia e quindi di anlizzare come si sia evoluta la situazione nella nostra regione nel corso degli anni dal 2012 ad oggi, cercando di capire quali siano stati i provvedimenti e gli eventi più efficaci nell'abbattimento delle concentrazioni.  
Nel capitolo 2 viene introdotto random forest e viene mostrato come questa tecnica possa esssere utilizzata per eliminare gli effetti della meteorologia e della stagionalità dall'andamento delle concentrazioni.  
Nel capitolo 3 viene analizzata, utilizzando la tecnica appena discussa, la situazione regionale andando a controllare l'andamento delle serie dei principali inquinanti nei capoluoghi di provincia lombardi.  
Nel capitolo 4 verranno presentate le analisi svolte sull'influenza del traffico sulle concentrazioni di tre inquinanti: ossidi di azoto, particolato atmosferico (PM10 e PM2.5) e monossido di carbonio.  
Nel capitolo 5 verrà presentata un'analisi del periodo di diffusione dell'epidemia di Covid-19 e di quali siano stati gli effetti del conseguente lockdown svoltosi nei mesi di marzo ed aprile sulle concentrazioni.

\chapter{Normalizzazione meteorologica utilizzando Random Forest}
%QUESTO CAPITOLO È IN FASE DI RISCRITTURA, POICHÈ CI TENGO CHE SIA MOLTO CHIARO E PRECISO NELL'ILLUSTRARE LA NOSTRA TECNICA E NELLO SPIEGARE LA SUA CORRETTEZZA
L'atmosfera è lo strato gassoso che ricopre esternamente la terra come un involucro ed è costituita principalmente da azoto e ossigeno, mescolati con piccole quantità di altri gas e pulviscolo. La porzione più bassa, che sta a diretto contatto col suolo, è definita strato limite planetario (o, in inglese, Planetary Boundary Layer) e sia la sua struttura che l'evoluzione hanno influenze fondamentali sul comportamento degli inquinanti atmosferici di maggior interesse.\cite{dina2009concentrazione}  
A causa della sua struttura, infatti, gli inquinanti emessi in atmosfera rimangono bloccati al suo interno e variazioni sull'altezza, che può variare da qualche centinaio di metri a qualche chilometro, causate da agenti atmosferici come il vento, la temperatura e la pressione atmosferica, hanno l'effetto di disperdere o aumentare le concentrazioni di inquinanti.\cite{stull2012introduction}  
La forte variabilità stagionale che si riscontra su tanti inquinanti è infatti direttamente collegata al fatto che nei mesi invernali (così come nelle ore notturne) l'altezza dello strato di rimescolamento sia ridotta e porti quindi ad un aumento dele concentrazioni.  
Chiaramente per poter analizzare i dati sull'inquinamento atmosferico volendo capire come le misure messe in atto nel corso degli anni abbiano influenzato l'andamento delle concentrazioni non possiamo trascurare questa grossa influenza da parte della meteorologia e del clima, altrimenti non sarebbe possibile stabilire con certezza quanta influenza possa avere avuto un determinato provvedimento rispetto a quella data dalla variazione delle condizioni atmosferiche.  
Per poter provare a capire ed eliminare l'influenza delle condizioni atmosferiche dalle concentrazioni di inquinanti vogliamo creare un modello statistico che, usando variabili che sappiamo hanno un'influenza sugli inquinanti, sia in grado di predirre con una buona precisione la concentrazione dell'inquinante che probabilmente si sarebbe misurata con tali le condizioni descritte dai valori delle variabili.  
Le variabili che noi vogliamo usare per fare previsioni sugli inquinanti, chiamate infatti predittrici, saranno le misurazioni di fenomeni atmosferici come il vento, le precipitazioni, la temperatura, la pressione atmosferica e la radiazione solare. Inoltre useremo anche il giorno dell'anno, per modellare la variabilità stagionale caratteristica di molti inquinanti, e la data in modo da poter tenere conto di eventuali trend presenti nell'andamento delle concentrazioni. %RIVEDI  
Il campo del machine learning, una fusione tra statistica, data science e informatica, viene in nostro soccorso per la creazione di questo modello, poichè ci sarebbe altrimenti impossibile riuscire a capire come ciascuna variabile possa direttamente influenzare la concentrazione dell'inquinante.  
Da un punto di vista matematico, infatti, il fatto che la concentrazione di un inquinante (che chiameremo $Y$), sia calcolabile a partire da un set di valori (che chiameremo $X = (X_1, X_2, .., X_n)$), può essere definito come:  
\[Y = f(X) + \epsilon\]
dove $f$ è una funzione fissata ma a noi sconosciuta ed $\epsilon$ è un termine d'errore casuale con media 0.  
Il machine learning ci permette, studiando un set di dati realmente raccolti sia della variabile da predirre che dei diversi predittori, di creare modelli statistici che provano a capire l'andamento di $f$ con la maggior precisione possibile.  
Chiaramente poi, avendo a disposizione un modello di cui ci fidiamo con una certa confidenza, possiamo andare poi a normalizzare la serie delle concentrazioni di un inquinante facendo fare previsioni al nostro modello per ogni data utilizzando condizioni meteorologiche medie. %RIVEDI!
Random forest è un metodo di machine learning d'insieme molto popolare, inventato da Leo Breiman nel 2001\cite{breiman2001random}, basato sull'uso di molti alberi di decisione ed adatto ad affrontare sia problemi di classificazione che di regressione, che sono quelli a cui siamo interessati nel nostro caso visto che vogliamo prevedere una quantità.

% Random forest
% - cenni storici
% - vantaggi di random forest
%	- no overfitting
% - perchè proprio questo
%		- semplicità
%		- non preoccuparsi di correlazione ecc
%		- performance
%		- ready to use con python
%
% Open data Lombardi
% - fondamentale per questo tipo di tecnica avere a disposizione una buona mole di dati su cui andare a creare i nostri modelli
%	
% Normalizzazione meteorologica
% - quindi, come sono state ottenute le serie normalizzate?

\chapter{La situazione in Lombardia}
Per prima cosa sono state affrontate una serie di analisi in cui, tramite l'applicazione di random forest per ottenere la normalizzazione dei dati relativamente alle condizioni meteorologiche (come descritto nel notebook RandomForest.ipynb), si è andato a delinare il quadro della situazione per quanto riguarda gli inquinanti atmosferici in Lombardia mostrando anche come questa tecnica funzioni e quali risultati si ottengano dal suo utilizzo.

Sono quindi state analizzate le serie storiche normalizzate dei principali inquinanti atmosferici, usando i dati registrati nei dodici capoluoghi di provincia lombardi. Per ogni inquinante di ogni capoluogo è stato creato un modello apposito, così da poter analizzare ogni situazione con un modello specifico creato appositamente per cercare di stimare al meglio le dinamiche locali degli eventi climatici e la loro relazione con gli inquinanti considerati. Chiaramente quindi per poter applicare la tecnica oltre alle registrazioni delle concentrazioni dell'inquinante sarà necessario recuperare anche i dati delle rilevazioni per le variabili meteorologiche che coinvolgeremo nel nostro processo.

Rispetto a quanto fatto in precedenza, a partire da queste analisi il metodo di preparazione dei dati su cui si va a creare il modello è stato leggermente modificato. Prima, infatti, venivano considerate tutte le date per cui risultava disponibile la rilevazione dell'inquinante e se mancavano i valori delle altre variabili questi venivano riempiti con la media calcolata su tutte le osservazioni. Così facendo, però, se uno dei sensori relativo a queste variabili era stato messo in funzione tempo dopo rispetto a quello dell'inquinante le rilevazioni di tutte le date antecedenti venivano riempite col valore medio. Questo portava quindi ad avere dei dati meno "realistici" ed anche i modelli ottenuti risultavano meno precisi. Ora invece si considerano solo le osservazioni successive alla data di messa in funzione del sensore più recente tra tutti quelli coinvolti, andando a riempire con i valori medi solo eventuali buchi isolati dovuti a malfunzionamenti dei sensori di rilevamento. Questo ci permette sicuramente di ottenere quindi dei modelli più precisi, ma al prezzo di dover ridurre il periodo delle nostre anlisi in base alla disponibilità dei dati.

I modelli ottenuti hanno avuto tutti prestazioni molto buone, specialmente quelli relativi all'ozono, con le polveri sottili che invece si sono rivelate l'inquinante a cui i nostri modelli fanno più fatica ad adattarsi, evidenziando come per questo tipo di inquinanti le sole variabili delle rilevazioni meteorologiche di superfice e temporali considerate non siano sufficienti per spiegarne completamente l'andamento, che è governato da eventi su scale molto più grandi di quelle descritte dalle nostre variabili. 

Nei modelli ottenuti la variabile più importante è stata la temperatura, seguita dalla data e dal giorno dell'anno. Questa considerazione però non dovrebbe meravigliarci se pensiamo al collegamento che esiste tra ciascuna e l'inquinamento: la temperatura serve sia ad identificare i cicli stagionali che eventuali periodi più caldi o freddi della media in cui la gente potrebbe inquinare maggiormente in risposta a questi eventi, la data ci permette di identificare l'evoluzione nel corso degli anni dei valori registrati, e quindi la presenza di eventuali trend o andamenti particolari, e il giorno dell'anno serve per identificare la stagionalità caratteristica degli inquinanti.  
Ancora una volta, invece, sembra che alle due variabili legate alle precipitazioni venga data una scarsa importanza. È bene ricordare, però, che l'importanza delle variabili è calcolata come il totale della diminuzione dell'impurità dei nodi della foresta, pesata per la probabilità di raggiungere ciascun nodo. Questa è quindi una misura che cerca di stimare come lavori il nostro modello per fare previsioni, ma non indica assoultamente che tipo di relazioni ci siano tra la variabile considerata e quella da predirre. Quindi, anche se nel nostro modello alle precipitazioni viene associata una bassa importanza, non ci sta dicendo che non abbiano influenza sulle concentrazioni registrate, ma semplicemente che la variabile per lui è meno utile per fare previsioni più precise. %RIVEDI, AMPLIA DESCRIZIONE IMPORTANZA

\section{NOx e NO2}
Nella categoria degli ossidi di azoto sono considerati due inquinanti di diverso interesse e che quindi è bene analizzare separatamente. Il monossido di azoto, infatti, alle concentrazioni tipiche misurate non risulta pericoloso nè per la salute umana nè per la vegetazione; il biossido di azoto, invece, desta sicuramente una maggior preoccupazione per i suoi effetti sulla salute. Per questo motivo considererò ed analizzerò separatamente le serie dei sensori misuratori di NOx e di quelli dedicati solo al biossido.%aggiungi citazione

Inizialmente abbiamo quindi analizzato le serie degli ossidi di azoto. Quando si considerano questi inquinanti è bene ricordare che il monossido ha principalmente origine primaria, mentre il biossido si genera in atmosfera grazie all'ossidazione del primo e quindi ha origine principalmente secondaria. All'emissione, infatti, si stima che il biossido sia circa il 5/10\% del totale per questa categoria.\cite{arpa2018rapporto}  
Entrambi sono emessi in atmosfera da processi di combustione ad alta temperatura (impianti di riscaldamento, motori dei veicoli, combustioni industriali, centrali di potenza, ecc..) e per ossidazione dell'azoto presente in atmosfera (o dei suoi composti contenuti nei combustibili utilizzati). Il biossido di azoto, oltre che alla pericolosità per quanto riguarda la salute di persone e vegetazione, ha un ruolo fondamentale nella formazione dello smog fotochimico in quanto è l'intermediario per la produzione di inquinanti secondari come l'ozono.

Secondo l'inventario regionale INEMAR 2017 la categoria maggiormente responsabile per le emissioni di questi inquinanti è il trasporto su strada, con un contributo pari al 50\% del totale delle emissioni, seguita dalle combustioni industriali e civili (con circa il 15\% di contributo ciascuna). Sempre secondo l'inventario il combustibile maggiormente responsabile delle emissioni di questa categoria è il diesel, con oltre il 50\% del totale delle emissioni, seguito dal gas naturale (20\%).

Le prime normative introdotte per contenerne le concentrazioni sono arrivate nei primi anni 90, imponendo limiti alle emissioni degli impianti e favorendo l'uso del gas naturale al posto di gasolio e cherosene come combustibili. Negli anni più recenti i limiti normativi sono stati ulteriormente abbassati. Per quanto riguarda il traffico, visto che è il settore maggiormente responsabile, i provvedimenti più importanti sono sicuramente l'introduzione delle marmitte catalitiche e l'introduzione di limiti alle emissioni dei veicoli, con le famose categorie EuroX. Proprio questo ultimo aspetto ha fatto sì che, sebbene nel corso degli anni il numero dei veicoli circolanti e dei kilometri percorsi sia aumentato, l'inquinante abbia comunque fatto registrare un trend in forte calo nel corso degli anni.

Nei modelli ottenuti le variabili più importanti sono state la temperatura, che è risultata quella col valore decisamente maggiore rispetto alle altre, il giorno dell'anno e la velocità del vento. Sicuramente non siamo sorpresi dalle prime due, visto il classico andamento stagionale dell'inquinante, legato sia alle maggiori emissioni antropiche durante il periodo invernale che alle condizioni più sfavorevoli alla dispersione. Anche la velocità del vento non ci stupisce, vista appunto la forte azione dispersiva svolta da questo elemento.

%inserisci immagini grafici

Innanzitutto controllando il grafico delle medie mensili si nota come la serie normalizzata riesca a non essere influenzata dall'andamento stagionale, mantenendo sempre un livello "medio" costante, che è proprio il risultato che ci aspettavamo di ottenere dall'applicazione della nostra tecnica. Noi volevamo riuscire ad avere delle serie che fossero indipendenti da tutti questi fattori che possono incidere sulle concentrazioni e visto l'andamento mostrato da quella ottenuta, che rimane approssimativamente intorno ad un valore "medio" indipendentemente dalla giornata, possiamo essere abbastanza fiduciosi di essere riusciti a stimarla con una buona precisione.
Sulle medie annuali vengono sicuramente eliminati i picchi, probabilmente causati da condizioni meteorologiche più favorevoli/sfavorevoli all'accumulo avute da alcuni anni in particolare (ad esempio 2006 o 2011), ma l'andamento delle due serie risulta comunque compatibile, ancora una volta a dimostrazione della validità del risultato ottenuto, che sarebbe sicuramente stata messa in discussione se si fossero ottenuti dei risultati completamente diversi.

Guardando i grafici si vede chiaramente che l'andamento delle concentrazioni misurate dal 2006 ad oggi sia abbastanza costante, con un aumento abbastanza significativo registrato tra gli anni 2012 e 2016, seguito però poi da un costante calo che ormai le ha riportate ai livelli di 8 anni fa. Se avessimo avuto a disposizione le serie complete a partire dai primi anni 90 sicuramente avremmo visto un trend negativo nelle concentrazioni, poichè è stato proprio in quel decennio che le emissioni sono state maggiormente ridotte. Negli ultimi anni non sono state introdotte nuove importanti normative a riguardo, se non aggiornamenti dei limiti imposte da quelle già esistenti (sia per quanto riguarda le emissioni degli impianti che dei veicoli). Questo sicuramente ha aiutato a mantenere le concentrazioni su un livello abbastanza costante, nonostante il numero di veicoli circolanti sia in continuo aumento.

Successivamente ci siamo occupati del biossido di azoto misurato singolarmente. Per le concentrazioni di questo inquinante il D.Lgs 155/2010 ha stabilito le seguenti soglie: 200$\mu g/m^3$ sulla media oraria, con un massimo di 18 superamenti annui concessi, e di 40$\mu g/m^3$ per la media annua. Se il limite sull'ora non viene praticamente mai superato, quello della media annua non sempre viene rispettato. Nel rapporto di ARPA Lombardia\cite{arpa2018rapporto} per quanto riguarda la provincia milanese, nel 2018, 7 stazioni su 14 hanno fatto registrare una media annuale superiore al limite imposto.

È chiaro quindi che questo sia un inquinante di interesse, soprattutto per quando si vuole analizzare l'impatto del traffico sulla qualità dell'aria, essendone così fortemente responsabile.

%inserisci immagini grafici

Anche per il biossido considerato separatamente le serie ottenute presentano un andamento abbastanza costante e anche in questo caso sembra che dal 2016 le concentrazioni siano leggermente in calo, anche se già in passato (2010-2013) si erano verificati andamenti simili, salvo poi essere tornate ad aumentare negli anni successivi.
Sicuramente credo sia importante notare che le medie annuali delle serie normalizzate rimangano sempre sotto alla soglia imposta dalla normativa vigente, a segno che comunque la situazione per quanto riguarda questo inquinante non sia proprio così critica come viene invece descritta a volte. Chiaramente questo non significa che non sia importante monitorarla ed occuparsene, visto che comunque in casi di particolari condizioni atmosferiche può capitare che si abbiano delle situazioni leggermente problematiche, ma di certo le concentrazioni di questo inquinante non possono essere considerate un problema grave attualmente.
C'è inoltre da dire che nel corso dei prossimi anni questo inquinante potrebbe ulteriormente calare, poichè l'innovazione tecnologica, soprattutto per quanto riguarda il mondo delle auto e il rinnovo della flotta circolante porteranno ad avere ulteriori abbassamenti delle emissioni. Chiaramente ci si può aspettare di vedere un processo simile anche per quanto riguarda le combustioni industriali e civili, che sono gli altri due settori importanti per quanto riguarda questo inquinante.

\section{PM10 e PM2.5}
Le polveri sottili sono sicuramente l'inquinante più discusso degli ultimi anni essendo quello le cui concentrazioni fanno registrare il maggior numero di superamenti della soglia imposta per legge e del quale è noto l'elevato impatto ambientale e sulla salute degli esseri viventi. Questo fa sì che si accendano spesso dibattiti sulla qualità reale dell'aria che respiriamo e su quali misure siano necessarie per riuscire a mantenere le concentrazioni sotto al valore limite oltre il quale il WHO ha riconosciuto la possibilità di avere danni alla salute.\cite{world2016air} Prima di andare ad analizzare i dati è però bene aver chiaro il contesto in cui si svolgeranno queste analisi. Il particolato è sicuramente un inquinante molto legato alla stagionalità, sia perchè le condizioni meteorologiche invernali sono più favorevoli all'accumulo rispetto a quelle estive che per le maggior emissioni antropiche tipiche della stagione, causate, ad esempio, dall'uso dei riscaldamenti. Va inoltre ricordato che la Lombardia, come tutto il bacino padano in generale, si trova in una posizione geografica sfavorevole, che porta alla maggior formazione di accumuli, soprattutto durante il periodo invernale.

Il particolato classicamente viene diviso in due categorie: PM10 e PM2.5, a seconda del diametro aerodinamico delle particelle esaminate. Quindi particelle con questo diametro, che non si riferisce alla dimensione della particella ma alla sue caratteristiche aerodinamiche, al di sotto dei 10$\mu$m sono classificate come PM10 e il discorso analogo vale anche per il PM2.5. Essendo appunto particelle molto piccole risultano quindi pericolose per la salute poichè riescono a penetrare più a fondo nell'apparato respiratorio, arrivando quindi ad arrecare danni maggiori. La pericolosità del particolato non deriva solamente dalla sua composizione, che è molto varia per le diverse origini che può avere, ma anche dal fatto che fa da veicolante per altri inquinanti più pericolosi, che si legano in atmosfera e vengono poi trasportate all'interno del corpo dalla particella. Questo inquinante ha appunto origini molto varie, sia di tipo primario da attività come le industrie, i riscaldamenti, il traffico e le combustioni in generale, ma anche di tipo secondario, poichè in atmosfera può formarsi a seguito di trasformazioni chimico-fisiche di altre sostanze.

Secondo INEMAR 2017 la categoria maggiormente responsabile per le emissioni di questi inquinanti sono le combustioni non industriali, quindi, ad esempio, i riscaldamenti domestici, specialmente quelli a biomasse. Il contributo di questa categoria si aggira intorno al 45\% per quanto riguarda il PM10 ed al 50\% per il PM2.5. Al secondo posto troviamo il traffico, con un contributo che arriva al 25\% per quanto riguarda la frazione più grossa del particolato e del 20\% per la più piccola. Per quanto riguarda i combustibili, il legno è di gran lunga il maggior responsabile delle emissioni di particolato, con oltre il 50\% del totale. Il diesel, che spesso viene adirittura accusato di essere il maggior responsabile di questo tipo di emissioni, si ferma invece a solo circa il 10\%.

Il PM10 è regolato dall'inizio degli anni 90, con i decreti già citati per gli ossidi di azoto che hanno introdotto limiti e nuove regolamentazioni per le emissioni degli impianti.Per quanto riguarda il settore del traffico l'introduzione più importante è stata quella del filtro antiparticolato nelle marmitte, che permette di ridurre drasticamente le emissioni, insieme chiaramente alle progressive limitazioni imposte tramite le categorie EuroX. Quando si parla di traffico e polveri sottili è importante ricordare che i motori a benzina non generano questo inquinante ed infatti le loro emissioni non sono normate nemmeno dalle sopracitate categorie internazionali.

Per quanto riguarda il PM10 il D.Lgs 155/2010 impone questi limiti di legge: 50$\mu g/m^3$ come media giornaliera, che non deve essere superata in più di 35 occasioni, e 40$\mu g/m^3$ per la media annuale. Un aspetto su cui potrebbe valer la pena indagare è il motivo per cui, per questo inquinante, il limite per la media giornaliera sia così vicino a quello per la media annuale, mentre per altri inquinanti questa differenza è molto più ampia.

Nei modelli ottenuti per trattare questo inquinante la variabile più importante risulta essere, non sorprendentemente vista la sua azione dispersiva, la velocità del vento. Al secondo posto troviamo la temperatura ed al terzo il giorno dell'anno, che sono collegate al classico andamento stagionale che fanno registrare le concentrazioni di questo inquinante. Vediamo inoltre come al numero di giorni dalle ultime precipitazioni registrate venga data un'importanza abbastanza discreta; neanche questo dovrebbe sorprenderci però, infatti è noto come la pioggia abbatta le concentrazioni portando al suolo parte del particolato presente in atmosfera e la sua assenza prolungata, soprattutto in periodi come quello invernale dove la dispersione naturale è resa più difficile dalle condizioni climatiche, non può quindi che portare ad un innalzamento. Grazie all'importanza ottenuta possiamo quindi capire che i modelli costruiti sono riusciti in qualche modo a rilevare questo tipo di relazione e perciò sono anche in grado di poterne eliminare gli effetti più efficacemente.

%inserisci grafici

Nei grafici ottenuti è evidente la presenza di un trend in costante calo, anche se negli ultimi anni sembra aver rallentato molto ed i valori rimangono piuttosto costanti.
Importante notare come successivamente al 2012 le concentrazioni normalizzate siano rimaste sempre sotto al limite per la media annuale di 40$\mu g/m^3$, sia per quanto riguarda le medie annuali che per quelle mensili.
Ancora una volta vediamo come le serie normalizzate ottenute mantengano un andamento medio rispetto ai dati grezzi, a conferma nuovamente della validità dei risultati ottenuti tramite questa tecnica.
Guardando i grafici sembrerebbe quasi che nel corso degli ultimi anni si possa aver raggiunto una sorta di limite, dovuto dalla quantità di attività antropiche (e in parte probabilmente anche da un fondo naturale) presenti nella nostra regione, oltre il quale sarà più complicato scendere ulteriormente. Quello che invece risulta evidente è che, se non fosse per le condizioni sfavorevoli in cui si trova la Lombardia e che quindi nei periodi più critici portano ad avere episodi con concentrazioni più alte, la qualità dell'aria anche per quanto riguarda questo inquinante non è assolutamente così terribile come spesso invece si sente dire.

Allo stesso modo abbiamo trattato il PM2.5, per il quale ricordiamo che il limite di legge è fissato a 25$\mu g/m^3$ sulla media annua, introdotto per la prima volta a partire dal D.Lgs 155/2010.

%inserisci grafici

Anche per questo inquinante viene evidenziato un trend negativo leggermente minore di quanto riscontrato col PM10 e che però negli ultimi anni presenta la stessa tendenza al rallentamento, che rinforza i nostri sospetti di aver raggiunto una sorta di limite.
Notiamo come dal 2015 le concentrazioni normalizzate siano riuscite anche a scendere sotto al limite di legge imposto e come, anche in questo caso, siano fortemente influenzate dalle condizioni sfavorevoli che poi portano spesso ad avere episodi problematici.

\section{CO}
Il monossido di carbonio è un inquinante molto pericoloso per la salute, in quanto è nota la sua migliore capacità di legarsi all'emoglobina rispetto all'ossigeno, causando notevoli danni per l'uomo.
È un inquinante che viene prodotto da combustioni in difetto di ossigeno ed ha origine prevalentemente primaria, essendo emesso direttamente da tutti i processi di combustione incompleta dei composti carboniosi (gas naturali, propano, carburanti, benzine, carbone, legna, ecc..).

Secondo quanto riportato da INEMAR 2017 i due settori maggiormente responsabili delle emissioni di questo inquinante sono il traffico ed i riscaldamenti domestici e le combustioni non industriali, con un contributo pari rispettivamente al 38\% e 27\% del totale delle emissioni annue. Le emissioni collegate al settore dei trasporti hanno fatto registrare forti cali a partire dagli anni 90, con una riduzione vicina addirittura al 90\% favorita da innovazioni quali le marmitte catalitiche e in generale il progresso tecnologico che, per esempio, ha permesso di passare dai 12,66gCO/km emessi dai motori a benzina classificati come Euro 0 (quelli antecedenti all'istituzione delle normative europee) a 1gCO/km che viene imposto come limite per la categoria Euro 6.
Per quanto riguarda invece il settore dei riscaldamenti domestici nel corso degli ultimi anni le emissioni hanno seguito un trend sempre crescente, causato soprattutto dall'uso della legna come combustibile, che però porta effetti riscontrabili solo localmente dove è più utilizzata (e solitamente le aree cittadine non sono questo tipo di zone).

Il limite di legge alle concentrazioni di questa sostanza in atmosfera è fissato a 10 $\mu g/m^3$ a partire dal 2005, anche se ormai sono più di vent'anni che non vengono registrati valori maggiori. Il monossido, quindi, ormai non rappresenta più un problema e le sue concentrazioni in atmosfera sono molto vicine al fondo naturale (come citato da ARPA nel rapporto sulla qualità dell'aria di Milano 2018) e spesso si arriva anche ai limiti della rilevabilità da parte dei sensori.
Pur non essendo più un problema abbiamo ritenuto che potesse comunque valer la pena di analizzare la situazione anche per quanto riguarda questo inquinante, ripetendo quanto fatto in precedenza per gli altri.

%inserisci grafici

Dal 2006 ad oggi le concentrazioni di monossido di carbonio sono rimaste piuttosto stabili, mostrando un andamento abbastanza costante, con un leggero aumento tra il 2010 e il 2014, seguito poi però da una fase di leggero calo. Questo andamento, che risulta praticamente costante e caratterizzato da variazioni piuttosto casuali, può sicuramente farci pensare di essere vicini alla misura del fondo naturale anche per questo inquinante, che ormai da anni non rappresenta più un problema.
Va infatti notato come nel periodo preso in esame i valori siano sempre rimasti sotto al 1 $\mu g/m^3$, ben più bassi del limite imposto dalla legge. Inoltre le medie delle serie normalizzate seguono abbastanza bene quelle delle concentrazioni reali, facendoci quindi pensare che le fluttuazioni possano essere più oscillazioni nel valore del fondo naturale che cambiamenti dovuti ad attività umane o altri fattori come quelli considerati nel nostro modello.

\section{Ozono}
L'ozono troposferico è un inquinante di origine secondaria, che si forma in atmosfera quando, favoriti da alte temperature e forte irraggiamento, ossidi di azoto e composti organici volatili subiscono trasformazioni chimico-fisiche che portano alla sua formazione. Per questo motivo l'ozono è considerato smog fotochimico ed il periodo critico per le sue concentrazioni, a differenza degli altri inquinanti, è quindi l'estate.%aggiungi cit?
I precursori dell'ozono arrivano generalmente da combustioni civili ed industriali e da processi che usano o producono sostanze chimiche volatili, ma avendo una formazione più complessa è chiaramente più difficile capire quali siano le sue origini e quindi dove intervenire per controllarne efficacemente le concentrazioni.

L'ozono è noto per la sua pericolosità per l'apparato respiratorio, al quale può causare problemi temporanei anche a seguito di un'esposizione a basse concentrazioni; chiaramente se l'esposizione avviene in modo ripetuto o prolungato il rischio di avere dei danni permanenti aumenta. In generale comunque, soprattutto ad alte concentrazioni, l'ozono diventa pericoloso sia per gli esseri umani che per la vegetazione, sulla quale può avere effetti negativi.

La normativa attuale fissa a 120 $\mu g/m^3$ il valore obbiettivo per la media mobile calcolata su 8 ore, con un massimo di 25 superamenti annui calcolati sulla media di 3 anni. Attualmente questo limite si fa fatica a rispettarlo (nel 2018 nella provincia di Milano tutte le stazioni l'hanno abbondantemente superato), ma ciò comunque non costituisce una criticità per la regione, in quanto questi superamenti sono tutti collegati al classico andamento stagionale.

L'andamento delle concentrazioni a partire dal 2013 è leggermente in aumento, anche se si mantiene su un livello piuttosto costante. Sicuramente questo è un inquinante su cui è difficile intervenire, per via delle sue origini complesse, perciò in futuro andrà mantenuto monitorato il suo andamento per verificarne l'evoluzione.

Un aspetto da notare è che l'ozono, tra tutti gli inquinanti, è quello che la nostra tecnica riesce a trattare con la maggiore precisione ed infatti i modelli ottenuti usati per ricavare le serie normalizzate sono quelli che in fase di costruzione hanno mostrato le performance migliori. Questo probabilmente è collegato alla forte relazione che questo inquinante ha con la temperatura e, soprattutto, la radiazione solare, che i nostri modelli hanno ben individuato (come si è visto con l'importanza delle variabili) e che quindi gli permetteva di essere più precisi nell'attività predittoria.

\section{Ammoniaca}
L'ammoniaca è un inquinante prodotto da processi degradativi di sostanza organica, le cui principali sorgenti sono infatti attività agricole e, in misura minore, il traporto su strada, lo smaltimento dei rifiuti e la combustione di legna e combustibili fossili. È un gas molto solubile in acqua, alla quale conferisce basicità e, tramite l'azione di particolari batteri, può favorire l'acidificazione del suolo. Inoltre l'interesse per l'ammoniaca è sempre crescente a causa della sua partecipazione nella formazione di particolato secondario quando presente in atmosfera.

Secondo INEMAR 2017 in Lombardia il 96\% delle emissioni di questa sostanza sono collegabili al settore dell'agricoltura, che infatti è il principale responsabile per quanto riguarda questo inquinante. Il traffico, invece, arriva a dare un contributo pari solamente al 1\% del totale annuo.

La normativa attuale non impone limiti per le concentrazioni registrate, anche se una serie di direttive europee per lo sviluppo rurale ha mirato, nel corso degli anni, a favorire la diffusione di buone pratiche per contenerne le emissioni, come ad esempio il divieto dello spargimento di liquami. 

Nei modelli ottenuti la variabile più importante è risultata essere la data, seguita dal giorno dell'anno e dalla temperatura, con la prima a cui è stata attribuita un'importanza decisamente maggiore rispetto a tutte le altre. Questo di solito avviene in due possibili scenari: o l'andamento presenta un trend e quindi la data serve proprio per identificarlo al meglio oppure è abbastanza casuale e la data diventa il modo migliore per capirne l'evoluzione.

%inserisci grafici

Vediamo come nel corso degli ultimi anni le concentrazioni si siano tenute su un livello abbastanza costante, con qualche piccolo aumento e diminuzione poco significativi. Direi che quindi ci troviamo nello scenario in cui il nostro modello usa la data poichè è la migliore informazione che riesce ad avere per stimare l'andamento abbastanza casuale, ma che si mantiene costante intorno ad un valore medio, delle concetrazioni di questo inquinante.%RIVEDI

\section{Benzene}
Il benzene viene sintetizzato dal petrolio e viene usato per produrre materie plastiche, resine sintetiche e pesticidi, come solvente e come sostanza antidetonante nella benzina. Il benzene, infatti, ad inizio anni 90 ha sostituito il piombo come additivo presente nella benzina, permettendo il passaggio da quella rossa a quella verde.
Quello presente in atmosfera deriva da processi di combustione incompleta di combustibili fossili, quindi da attività come traffico (soprattutto dai veicoli a benzina) e processi di combustione industriale.

La sua pericolosità per la salute umana varia molto a seconda della concentrazione e della durata dell'esposizione, ma anche alle basse quantità a cui viene misurato in atmosfera può comunque essere pericoloso: lo IARC (agenzia internazionale per la ricerca sul cancro) l'ha infatti inserito tra le sostanze per le quali esiste una sufficiente evidenza di cancerogenicità per l'uomo.

Il D.Lgs 155/2010 stabilisce per questo inquinante un valore limite di 5$\mu g/m^3$ per la media oraria, che viene ampiamente rispettato in tutta la regione. 

Nei modelli ottenuti l'importanza delle variabili si è rivelata essere abbastanza in linea con quanto visto per gli altri inquinanti, con temperatura, giorno dell'anno e data che risultano essere le più importanti.

%inserisci grafici

Per il benzene i maggiori cali si sono registrati a partire dai primi anni 2000, ed infatti nella parte iniziale del grafico si può notare maggiormente questa tendenza, che negli anni più recenti è andata a smorzarsi, probabilmente per l'avvicinamento a un valore limite oltre il quale sarà più difficile scendere ulteriormente.

\section{SO2}
Sebbene una volta fosse sicuramente uno degli inquinanti a destare più proccupazioni, così come è stato uno dei primi ad essere monitorato e limitato per legge, è già ormai diversi anni che le concentrazioni di biossido di zolfo sono ampiamente sotto al limite legislativo per questo inquinante (350 $\mu g/m^3$ media oraria), visto che a partire dagli anni 90 sono state introdotte una serie di normative atte a limitarne le emissioni. Le più importanti sono sicuramente quelle riguardanti gli usi di gasolio e nafta come forme di riscaldamento, che ormai sono stati completamente sostituiti dal metano, e quelle che nel corso degli anni hanno progressivamente abbassato il limite di zolfo che può essere contenuto nei carburanti (da 0.8\% nel 1980 a 0.2\% nel 1995 e infine a 0.1\% nel 2008).

Sebbene non vi siano più preoccupazioni per quanto riguarda le concentrazioni di questa sostanza in atmosfera abbiamo ritenuto che fosse utile verificare anche con essa i risultati ottenuti tramite la nostra tecnica. 

La data è risultata di gran lunga la variabile più importante, con giorno dell'anno e temperatura che sono le sole due variabili a cui viene data un minimo di importanza. Ancora una volta il fatto che la data sia così importante ci fa capire come i nostro modelli riescano ad identificare al meglio l'andamento casuale delle concentrazioni usando questa informazione.

%inserisci grafici

Come si può vedere le concentrazioni di SO2 registrate sono ormai a livelli bassissimi, tanto da non destare più preoccupazioni. L'andamento nel corso degli ultimi anni è abbastanza costante, fatto salvo per la classica variabilità stagionale, come si vede sia dalla serie normalizzata che già guardando i dati grezzi, e credo che anche i risultati ottenuti diano sostegno all'ipotesi di aver individuato il fondo naturale (confermato anche da ARPA, sempre nel rapporto sulla qualità dell'aria milanese).

\chapter{Effetti del traffico sui principali inquinanti atmosferici}
Avere a disposizione un modello che ci permette di eliminare l'influenza delle condizioni meteorologiche e della variabilità stagionale dalle concentrazioni degli inquinanti atmosferici ci permette di poter fare delle analisi più precise sull'efficacia di misure prese per contrastare l'inquinamento. L'andamento delle serie normalizzate, infatti, deve per forza essere causato da fattori non considerati nella loro generazione. Sicuramente i due più importanti sono: l'eventuale trend rilevato per le concentrazioni, che noi ovviamente non andiamo ad eliminare poichè è quello che ci interessa per monitorarne l'andamento, ed altri fattori che possono avere un'influenza e che non sono stati considerati, quindi ad esempio particolari situazioni in cui le emissioni antropiche sono state ridotte (COVID) o nuove normative messe in campo.

Quando si parla di inquinamento atmosferico uno degli aspetti su cui sicuramente il dibattito è più acceso e che attira più interesse è l'influenza del traffico veicolare sulla qualità dell'aria. Spesso infatti si sente additare il traffico come il principale responsabile delle concentrazioni degli inquinanti più preoccupanti (e, salvo gli ossidi di azoto, questa cosa non è vera) e sempre maggiori nel corso degli anni sono stati i provvedimenti presi nelle grandi città atti a ridurre il traffico, in nome di una migliore qualità dell'aria. Questi provvedimenti sono stati sempre molto contestati e dibattuti, poichè molti ritengono che il traffico non sia la causa principale dell'inquinamento e che la soluzione non sia bloccarlo ma studiare alternative per renderlo il più "efficiente" possibile (meno traffico, parcheggi, ecc..).

Il nostro obbiettivo è stato quindi di provare ad usare la nostra tecnica per verificare quali siano stati i risultati ottenuti in seguito all'applicazione di alcuni provvedimenti sul traffico presi nella città di Milano nel corso degli ultimi anni, in modo da mostrare come l'applicazione di tecniche per normalizzare i dati dell'inquinamento rispetto alla meteorologia permetta uno studio migliore dell'efficacia delle misure prese nell'abbattimento delle concentrazioni. Purtroppo la mancanza di dati relativi a velocità e direzione del vento, due misure (specialmente la prima) a cui di solito viene assegnata una discreta importanza dai nostri modelli, ci costringe a limitare le nostre analisi al periodo successivo al 2012. Cosi facendo possiamo quindi controllare l'andamento delle concentrazioni a partire dall'introduzione di Area C, anche se sarebbe stato interessante avere a disposizione anche dati degli anni precedenti, in modo da poter fare un confronto più esteso che coinvolgesse anche gli anni precedenti all'entrata in vigore del provvedimento.
Per verificare eventuali effetti derivanti dall'introduzione di questi provvedimenti siamo andati a confrontare l'andamento della serie normalizzata ottenuta per la stazione di Milano Via Senato, coinvolta appunto dal provvedimento Area C, con quelli delle serie di altre due stazioni: quella di Pioltello Limito, situata nell'hinterland milanese, e quella di Bormio, che si trova in un ambiente molto diverso da quello cittadino. La prima è stata scelta poichè, essendo comunque vicina alla città, la qualità dell'aria e le condizioni che la determinano dovrebbero essere abbastanza simili. Non essendo stata colpita da nessun provvedimento, inoltre, ci permette di avere una base di confronto per verificare se appunto l'applicazione di misure come Area C possa effettivamente aver portato a qualche risultato. La stazione di Bormio invece presenta caratteristiche completamente differenti, ma è comunque utile provare a verificare eventuali analogie o discrepanze tra le serie, per vedere quali siano le differenze tra due località così diverse.

\section{NOx e NO2}
\subsection{NOx}
Quando si parla di traffico e si vuole analizzare il suo impatto sulla qualità dell'aria gli inquinanti di maggiore interesse sono sicuramente gli ossidi di azoto, di cui questa categoria risulta responsabile per il 50\% delle emissioni secondo quanto riportato da INEMAR 2017.\cite{inemar2017}

Quando si parla degli ossidi di azoto relativamente al traffico l'attenzione viene posta maggiormente sul diesel, che è il maggior responsabile delle emissioni di questo inquinante. Nel corso degli ultimi anni il numero dei veicoli con motorizzazione diesel circolanti in Italia è andato sempre ad aumentare, tanto che IIR2020\cite{iir2020} stima un consumo quasi triplo di carburante per quanto riguarda veicoli diesel rispetto a quelli a benzina. Questo aumento di veicoli circolanti avrebbe dovuto aver l'effetto di innalzare le concentrazioni di ossidi di azoto misurate, ma è stato contrastato dall'innovazione tecnologica che è riuscita sempre di più a ridurre le emissioni prodotte (basti pensare alle categorie EuroX), introducendo importanti invenzioni come ad esempio la marmitta catalitica.

Quando avevamo analizzato in precedenza l'andamento della media tra le serie dei capoluoghi di provincia Lombardi avevamo ottenuto un andamento abbastanza costante nel corso degli ultimi 15 anni, con un leggero trend calante negli anni più recenti. Sembra quindi che anche per la Lombardia l'aumento dei veicoli circolanti motorizzati a diesel sia stato contrastato dal progresso tecnologico, che ha permesso di mantenere le concentrazioni sotto controllo.

Abbiamo quindi creato tre modelli per questo inquinante, uno per ogni località scelta per il nostro esperimento, e provato a verificare come negli anni si siano evolute le diverse situazioni.

I modelli ottenuti hanno mostrato tutti buone performance, che ci hanno consentito di affrontare le analisi potendoci fidare con una certa confidenza dei risultati ottenuti. Il modello di Bormio ha presentato una particolarità interessante: il valore dell'RMSE (l'errore medio commesso nelle previsioni del nostro modello) è molto minore rispetto a quanto ottenuto per le altre due località.

Innanzitutto abbiamo notato come per tutte e tre le località la velocità del vento sia una variabile a cui viene assegnata una buona importanza, se non addirittura la maggiore come nei casi di Limito e Bormio. Per quest'ultima stazione, addirittura, l'importanza risulta di gran lunga più alta rispetto a tutte le altre variabili. In realtà se pensiamo alle sue caratteristiche geografiche la cosa non dovrebbe sorprenderci più di tanto, poichè è evidente come in una valle montana l'azione del vento possa avere una funzione dispersiva ancora più importante che in altre località.
L'altra variabile che in tutti e tre i casi è stata riconosciuta come buon predittore risulta essere la temperatura, che può essere utile per verificare sia l'andamento stagionale che possibili eventi meteorologici particolari che possono portare ad un aumento delle emissioni antropiche (ad esempio periodi particolarmente freddi in cui vengono maggiormente usati riscaldamenti e automezzi rispetto a quando si registrano temperature più miti).  

%inserisci grafico
Guardando il grafico si nota innanzitutto come la serie di Bormio abbia davvero dei valori ridotti che ci mostrano come in località di questo tipo l'inquinamento da ossidi di azoto non sia sicuramente un problema.
Un altro aspetto che risalta subito all'occhio è l'andamento pressochè identico delle serie relative a Milano e Limito, che mostrano entrambe come a partire dal 2016/2017 si sia registrato un trend decrescente praticamente compatibile tra le due stazioni. Analizzando solo la serie di Milano avremmo potuto pensare che questo calo potesse essere collegato alla riduzione del traffico causata dai provvedimenti presi, ma osservare lo stesso andamento anche su una stazione diversa e non colpita da tali limitazioni ci indica che la causa deve per forza essere un'altra. Una possibile idea è che l'origine di questo trend sia proprio nell'innovazione tecnologica, i cui risultati stanno finalmente mettendosi in mostra nonostante le sempre maggiori attività umane, portando a registrare un calo generale e non specifico di alcune località. A Bormio, dove sicuramente la densità di attività antropiche inquinanti è nettamente ridotta rispetto alle altre due stazioni, questo calo risulta chiaramente molto minore, sia appunto perchè gli effetti dell'innovazione possono essere meno evidenti a causa della minor applicazione che perchè le concentrazioni sono già molto ridotte e quindi è ancora più difficile far registrare dei cali consistenti.

\subsection{NO2}
Per completezza è stato trattato anche il biossido di azoto, inquinante sempre molto collegato al traffico e attorno al quale c'è molto interesse per i possibili effetti sulla salute umana.

I modelli creati hanno mantenuto le buone performance viste con quelli creati per trattare gli ossidi ed anche l'importanza delle variabili è risultata praticamente invariata, non sorprendentemente vista la relazione tra le due misure.

%inserisci grafici

In questo caso notiamo una sostanziale differenza tra l'andamento delle serie di Milano e Limito. La prima, in modo abbastanza compatibile con quanto già visto per gli ossidi di azoto, mostra un trend decrescente a partire dagli ultimi cinque anni. Per la seconda, invece, tra gli anni 2015 e 2018 si è assistito ad un aumento delle concentrazioni, che prima si mantenevano su un livello abbastanza stabile, che sono tornare a riscendere nel corso degli ultimi due anni. È proprio negli stessi anni in cui la serie normalizzata fa registrare questo aumnento che proprio in quella zona è stata inaugurata l'autostrada BreBeMi, che potrebbe aver portato ad un aumento del flusso di traffico e conseguentemente delle concentrazioni registrate. Andrebbe però verificato per quale motivo a seguito del 2018 sembra che le concentrazioni tornino sul livello più o meno costante visto tra il 2012 ed il 2014.
Per quanto riguarda la situazione di Milano sembra che effettivamente la qualità dell'aria sia migliorata nel corso degli anni, potenzialmente anche grazie alla riduzione del traffico causata da Area C prima ed Area B poi. Se questa è sicuramente una potenziale causa è anche vero che probabilmente a contribuire ai ribassamenti c'è un trend di fondo, dettato da fattori come l'innovazione tecnologica, che infatti sembra parzialmente verificato anche per Pioltello (almeno a seguito del 2018) che per Bormio.

Nel caso degli ossidi di azoto, quindi, sembra che le misure prese a Milano abbiano portato a qualche miglioramento, specialmente per quanto riguarda il biossido, per il quale le serie normalizzate di Milano e Limito presentano andamenti diversi, al contrario di quello che si era visto trattando gli ossidi nel loro complesso.%RIVEDI

\section{CO}
Per quanto riguarda il monossido di carbonio INEMAR 2017 individua trasporto su strada e combustioni non industriali (quindi i riscaldamenti, in particolare quelli alimentati a biomasse) come le due principali fonti emissive, con un percentuale di circa il 60\% del totale.

La situazione per quanto riguarda questo inquinante non risulta essere critica, visto che le concentrazioni registrare sono ormai prossime a valori riconducibili al fondo naturale. Nonostante ciò abbiamo ritenuto fosse comunque utile fare un'analisi di questo tipo, sia per verificare eventuali effetti dei provvedimenti di limitazione al traffico presi, che per provare a cercare eventuali differenze tra le serie delle tre località. In questo caso risulta particolarmente interessante indagare anche su quella di Bormio, poichè è una zona in cui la legna viene ancora molto usata come forma di riscaldamento e quindi si possono indagare su quali siano gli effetti di questi "comportamenti".

La variabile con importanza maggiore nei tre modelli è stata la temperatura, che ci permette proprio di cogliere l'andamento tipicamente stagionale di questo inquinante. Inoltre, essendo derivante in buona parte dai riscaldamenti domestici, è abbastanza evidente come tale misura possa essere utilizzata per stimarne un loro utilizzo e quindi riuscire a fare previsioni più precise e che sappiano considerare anche questo aspetto.
La velocità del vento, chiaramente, ha ottenuto ancora una buona importanza, che ci conferma nuovamente la forte azione dispersiva di questo elemento.
Per quanto riguarda il modello di Bormio abbiamo visto che la radiazione globale risulta essere la variabile con importanza maggiore, al pari della temperatura. Anche per essa e per il collegamento con la presenza di sole (e quindi il possibile uso di riscaldamenti) e la stagionalità è naturale che valga lo stesso discorso fatto per la temperatura.

%inserisci grafico

Se le stazioni di Limito e Bormio presentavano valori già molto bassi e con un andamento costante, per la stazione di Milano possiamo vedere che nel corso degli ultimi anni si sia registrato un calo abbastanza importante delle concentrazioni. Attualmente i valori risultano ancora più alti rispetto a quelli delle altre due località, potenzialmente anche a causa della maggior densità abitativa (e quindi un maggiore uso di veicoli e riscaldamenti) della città rispetto alle altre due zone.
L'origine di questo calo potrebbe sicuramente essere in parte collegata i provvedimenti di limitazione del traffico, come potrebbe venire il sospetto guardando il grafico, poichè viene registrato solo per la stazione di Milano e non sulle altre due (a differenza di quanto visto in precedenza dove il calo risultava essere compatibile su tutte le località). Altre possibili ragioni potrebbero essere i provvedimenti di limitazione all'uso di riscaldamenti a biomasse nelle città presi nel corso degli ultimi anni, anche se il loro effetto probabilmente si sarebbe dovuto vedere, anche solo parzialmente, anche sulle serie delle altre località, in particolare quella di Limito che è stata soggetta alle stesse limitazioni.

\section{PM10}
Le polveri sottili sono sempre un inquinante molto discusso e che spesso viene (anche erroneamente) collegato al traffico. Secondo INEMAR 2017\cite{inemar2017} il traffico risulta essere responsabile di meno del 25\% delle emissioni di PM10, mentre il settore principale risultano ancora essere le combustioni non industriali, specialmente per quanto riguarda i riscaldamenti a biomasse. Avendo una composizione molto varia è comunque difficile stabilire in modo preciso l'origine del particolato atmosferico e quindi quali sono le fonti di maggiori emissioni.
Una cosa che però è bene ricordare quando si parla di polveri sottili e traffico è che gli unici responsabili di questo inquinante per questo settore sono i motori a diesel, poichè quelli a benzina non lo producono (ed infatti non viene limitato nemmeno dalle categorie EuroX).

Negli ultimi anni c'è stata molta attenzione riguardo a questo inquinante, specialmente per gli effetti che ha sulla salute umana.
Questo ha quindi portato anche molto interesse nella ricerca di soluzioni efficaci per il loro contenimento, che per quanto riguarda il traffico sono state individuate nell'uso di filtri antiparticolato che riducano le emissioni al tubo di scappamento. 

Anche in queste trattazioni si è riscontrato come la nostra tecnica abbia sempre prestazioni leggermente peggiori quando si vanno a trattare le polveri sottili, probabilmente a causa del fatto che le variabili da noi considerate non siano sufficentemente in grado di spiegare i fenomeni su più larga scala che influenzano le loro concentrazioni.

%inserisci grafici

Notiamo come l'andamento delle tre serie, sebbena quella di Bormio stia su valori molto minori, sia assolutamente compatibile con una grande similitudine tra quella di Milano e quella di Limito e vista la natura del particolato, che viene trasportato anche per grandi distanze, la cosa comunque non ci stupisce. In questo caso si vede come non si possa notare alcuna differenza tra gli andamenti delle serie che ci possa suggerire una reale efficacia dei provvedimenti presi in termini di limitazione del traffico. Anzi, guardando il grafico ottenuto, la serie di Limito (non coinvolto da nessuna misura) risulta calata leggermente di più rispetto a quella di Milano.
Il trend presentato è calante ma i valori sono sicuramente ancora molto vicini al limite imposto per legge e questa potrebbe essere un'indicazione della necessità di misure di diverso tipo per contrastare questo inquinante, che dovranno necessariamente coinvolgere i settori maggiormente responsabili delle emissioni e non solamente il traffico.

\section{Utilizzo di dati relativi al traffico}
L'algoritmo random forest ha il grande vantaggio di poter sempre provare ad introdurre nuove variabili predittrici per i nostri modelli senza doverci preoccupare di problemi come la correlazione o la collinearità, come invece succede con metodi come la regressione lineare. Infatti, in fase di costruzione, sarà proprio il modello stesso a scegliere quali variabili usare maggiormente per fare previsioni, scegliendo di volta in volta quelle che portano ad avere previsioni più accurate.

Per indagare ulteriormente sugli effetti del traffico sulle concentrazioni degli inquinanti abbiamo quindi provato ad introdurre una variabile che tracci appunto il numero di veicoli circolanti, in modo da costruire dei modelli che siano in grado di eliminare l'influenza del traffico dalle concentrazioni, in maniera analoga a quanto fatto con tutte le altre misure predittrici scelte.
Per fare questo sono venuti in nostro soccorso i dataset pubblicati dal comune di Milano con le registrazioni degli ingressi in AreaC. Tramite un apposito script abbiamo quindi recuperato i dati anche di tale dataset, organizzati in appositi file e usati poi nella preparazione dei dati su cui basare la costruzione del nostro modello. Cosi facendo abbiamo ottenuto un modello in grado di eliminare l'effetto del traffico dalle concentrazioni, restituendoci una serie che rappresenti quale sarebbe stata la concentrazione in una giornata con condizioni meteorologiche, stagionali e del traffico medie. Questa serie potrà poi essere confrontata con quelle generate dai modelli creati senza la variabile del traffico per cercare possibili discrepanze, che potrebbero appunto indicarci come l'effetto del traffico abbia pesato sulle concentrazioni registrate.

%inserisci grafico andamenti ingressi

Si nota come nel corso degli anni l'andamento sia sempre stato piuttosto costante, mostrando un andamento ciclico caratterizzato da un calo, prevedibile, nei mesi estivi.
Si può chiaramente notare l'effetto lockdown, che ha portato ad avere il traffico a livelli molto più bassi di quanto mai visto in precedenza.
Nel corso degli anni, fatta eccezione per la particolare primavera di quest'anno, l'influenza del traffico sulle concentrazioni dovrebbe essere rimasta piuttosto invariata, anche se l'ammodernamento della flotta circolante potrebbe sicuramente aver ridotto le emissioni totali, anche a parità di numero di veicoli circolanti.

\subsection{NOx}
Le prestazioni del modello ottenuto sono risultate in linea con quelle ottenute senza l'inclusione della variabile sul traffico. Questa misura, quindi, non ci permette di fare previsioni più precise di quanto non riuscissimo già a fare, ma ciò non significa che non ci possano essere delle differenze nelle serie normalizzate ottenute. 

Alla variabile è stata attribuita un'importanza media, che significa che viene abbastanza utilizzata per fare previsioni. L'importanza delle altre variabili del modello risulta invece praticamente invariata, con temperatura e velocità del vento che hanno continuato ad essere le due più importanti.

Uno strumento utile per verificare come il nostro modello utilizzi le variabili predittrici per fare previsioni sono i partial dependence plots. Come già visto in precedenza questi grafici ci mostrano come vari il valore delle previsioni fatte dal nostro modello al variare del valore di una singola variabile, con tutte le altre fissate al loro valore medio.
Analizziamo quindi il grafico riguardante la variabile del traffico per vedere come il nostro modello ne faccia uso.

%inserisci partial dependence plots

Non sorprendentemente, visto il forte collegamento tra questi inquinanti ed il traffico, si vede come le previsioni del nostro modello arrivino ad avere valori decisamente maggiori al crescere del numero di ingressi registrati, con un aumento di oltre 30$\mu g/m^3$ se si raddoppia il numero di ingressi.

%inserisci grafico 

A primo impatto si nota subito come le due serie siano sempre praticamente equivalenti, mantenendo lo stesso andamento. Si nota anche, però, come per il 2020 le due serie mostrino andamenti molto diversi, con una differenza di circa 10 $\mu g/m^3$ sulle concentrazioni previste.
Come avevamo visto in precedenza è proprio nel 2020 che il traffico ha fatto registrare un importante calo, dovuto alle misure per il contenimento dei contagi da COVID, e si nota chiaramente come in corrispondenza di questo calo le previsioni dei due modelli siano discordanti.
Da un lato si vede come il modello costruito senza l'uso del numero di ingressi registrati in Area C abbia effettivamente rilevato un calo abbastanza importante nelle concentrazioni misurate, dovuto proprio alla diminuzione del traffico. Dall'altro si vede invece come il modello costruito usando anche i dataset di Area C non rilevi questa differenza, continuando a mantenere un andamento in linea col trend decrescente di questi ultimi anni. Questo modello, infatti, essendo stato costruito utilizzando anche i dati sul traffico è capace di eliminarne l'influenza (e ciò avviene sia in condizioni di traffico al di sopra che al di sotto della media) e rapportando i dati dei mesi della primavera 2020 a condizioni di traffico medie ha eliminato il calo rilevato in precedenza. Questo ci dà ulteriormente conferma dell'origine da attribuire a tale calo ed inoltre ci dà una stima di come mediamente potrebbero cambiare le concentrazioni se venisse completamente (o quasi, proprio come è successo nei mesi di lockdown) eliminato il traffico veicolare.

Per quanto riguarda gli NOx abbiamo quindi dimostrato come effettivamente anche i nostri modelli siano in grado di individuare l'influenza del traffico sulle concentrazioni e abbiamo provato anche a quantificare il possibile miglioramento della qualità dell'aria che si otterrebbe eliminandolo completamente.
Un'altra nota da fare è che i modelli creati utilizzando anche i dati degli ingressi in Area C individuano comunque lo stesso trend decrescente nelle concentrazioni già visto in precedenza, mostrandoci quindi come questo andamento non sia associabile al volume del traffico, ma che le sue cause siano da ricercare altrove. Una delle possibilità più probabili è sicuramente che l'innovazione tecnologica, e quindi il calo delle emissioni, abbiano determinato questo calo.

\subsection{NO2}
Per completezza è stato trattato anche il biossido di azoto.

%inserisci grafico

Anche per questo inquinante valgono le stesse considerazioni fatte per gli ossidi. Importante notare come anche questa volta ci sia una differenza di andamento per l'anno 2020 associabile proprio al calo del traffico circolante dovuto al lockdown.

\subsection{CO}
Le stesse prove, effettuate con la creazione di un modello per la trattazione che faccia uso anche del numero degli ingressi in Area C, sono state fatte sul monossido di carbonio.

%inserisci partial dependence plot

Anche se l'aumento rilevato al raddoppiare del traffico non è sicuramente consistente come quelli riguardanti gli ossidi di azoto, anche in questo caso possiamo notare come il nostro modello associ una crescita delle concentrazioni al crescere del traffico. 

%inserisci grafico

Guardando il grafico si nota nuovamente come le serie ottenute siano sempre piuttosto equivalenti. Ci sono due piccole discrepanze sui risultati ottenuti, una riguardante il periodo dell'epidemia di COVID ed una a cavallo tra gli anni 2016 e 2017.
Per il periodo riguardante l'epidemia di COVID si nota la stessa situazione vista per gli ossidi di azoto, con il modello costruito senza i dati di Area C che fa previsioni leggermente più basse rispetto all'altro, potenzialmente collegabili alla diminuzione del traffico. Per quanto riguarda il periodo tra il 2016 ed il 2017 si vede invece come le previsioni di tale modello siano più alte, come se l'influenza del traffico in tale periodo sia stata maggiore. Controllando il grafico del numero di ingressi visto in precedenza, però, non si notano aumenti nel numero di veicoli circolanti che possano giustificare tale differenza, quindi le cause potrebbero essere altre oppure si potrebbe semplicemente trattare di una differnza delle previsioni dei due modelli che comunque può esserci, trattandosi di previsioni basate su modelli statistici e che quindi hanno una certa variabilità.
Trattandosi comunque entrambe di diffrenze minime è difficile stabilire con certezza le cause che stanno alla loro origine.

Per il monossido di carbonio, quindi, l'influenza del traffico sembra essere davvero minima, anche perchè si sta comunque parlando di concentrazioni molto basse. L'innovazione tecnologica del mondo delle auto ha fatto sì che questo inquinante ormai non sia più un effettivo problema.

\subsection{PM10}
Passiamo ora all'analisi dei risultati ottenuti facendo le prove viste in precedenza anche per le polveri sottili (in particolare il PM10 - per il PM2.5 si ha una situazione praticamente analoga).

%inserisci grafico

Vediamo infatti come le due serie rimangano praticamente sempre equivalenti, non mostrando particolari scostamenti neanche nel periodo dell'epidemia di COVID.

Per quanto riguarda le polveri sottili, quindi, il nostro metodo non rileva una particolare influenza dell'andamento del traffico sulle concentrazioni registrate, anche se rimane comunque un lievissimo trend negativo sulla serie.

\chapter{Effetti del lockdown per il Covid-19 sulle concentrazioni degli inquinanti}
I mesi primaverili del 2020 sono stati caratterizzati dalla diffusione dell'epidemia di Covid-19 e dalle importanti misure prese per il suo contenimento. Queste limitazioni hanno portato ad avere un quadro emissivo eccezionale per gli inquinanti atmosferici, che difficilmente si sarebbe potuto verificare in condizioni normali. Questi mesi, infatti, sono stati caratterizzati da importanti limitazioni sugli spostamenti e alle attività produttive, quindi rappresentano un importante banco di prove per verificare come siano cambiate le concentrazioni degli inquinanti in atmosfera con una così drastica riduzione delle attività antropiche, dandoci la possibilità di valutare come effettivamente si potrebbe intervenire per ottenere ulteriori miglioramenti della qualità dell'aria.

L'obbiettivo delle nostre analisi è stato mettere alla prova la nostra tecnica di normalizzazione per vedere come siano effettivamente cambiate le concentrazioni degli inquinanti durante i mesi dell'epidemia rispetto agli anni precedenti, dopo aver eliminato la variabilità derivata dalle condizoni meteorologiche e stagionali. È infatti importante considerare che l'epidemia è arrivata durante dei mesi che già normalmente sono abbastanza favorevoli per la maggior parte degli inquinanti di interesse, poichè le condizioni atmosferiche e meteorologiche dei mesi primaverili portano sempre ad un abbassamento delle concentrazioni misurate rispetto ai mesi invernali.
Potendo eliminare la variabilità data dalla stagionalità e da queste condizioni meteorologiche che influenzano così tanto i valori misurati è quindi sicuramente interessante cercare di capire quanto i classici cali delle concentrazioni registrati nei mesi primaverili siano stati influenzati dal blocco generale avuto dalle attività umane in tali mesi.

Abbiamo quindi trattato tutti gli inquinanti, confrontando l'andamento dei mesi tra febbraio e maggio del 2020 con quello degli anni tra il 2015 e il 2019 della media delle serie, normalizzate e non, dei capoluoghi di provincia già usate per le analisi precedenti. In questo modo potremo verificare se per i mesi dell'epidemia il nostro modello ha rilevato dei cali riconducibili proprio alle limitazioni imposte, in modo da capire quale tipo di interventi possa risultare più efficacie per il contenimento delle concentrazioni di ciascuno.

\section{NOx e NO2}
\subsection{NOx}
Abbiamo iniziato le nostre analisi trattando gli ossidi di azoto che, viste le limitazioni agli spostamenti e considerato il loro collegamento con l'andamento del traffico, potrebbero essere potenzialmente gli inquinanti più colpiti durante questi mesi di lockdown. Ad influire, inoltre, potrebbe sicuramente essere stato anche il parziale calo delle emissioni industriali, dovuto al fermo delle attività non ritenute essenziali.

%inserisci grafici

Guardando il grafico delle concentrazioni misurate si nota chiaramente il calo delle concentrazioni classico dei mesi primaverili, causato proprio dalle condizioni più favorevoli alla dispersione. Tale calo, chiaramente, risulta invece eliminato quando si considerano le serie normalizzate.
Guardando sia i grafici delle concentrazioni reali che quello delle serie normalizzate si osserva come nei mesi di lockdown le concentrazioni abbiano subito un importante ribasso rispetto agli anni precedenti, sicuramente dovuto al blocco del traffico e di alcune attività produttive. 

\subsection{NO2}
Abbiamo verificato anche i dati relativi al biossido di azoto, aspettandoci comunque di ottenre risultati abbastanza simili a quelli visti sugli ossidi trattati nel loro complesso.

%inserisci grafici

Anche in questo caso, infatti, oltre al classico calo dei mesi primaverili, si vede una differenza tra le concentrazioni del 2020 e quelle degli anni precedenti, sia considerando le concentrazioni misurate che guardando le serie normalizzate.
Anche in questo caso le cause sono sicuramente da attribuire alle stesse ragioni che hanno portato al calo degli ossidi.

Chiaramente una misura così restrittiva non è sicuramente applicabile in modo continuo, ma ci dimostra come interventi su larga scala possano avere un discreto impatto sulle concentrazioni a differenza di quello che si era visto analizzando dei provvedimenti presi in particolari località, i cui effetti si sono rivelati minimi se non addirittura nulli.
Interventi su larga scala richiederebbero investimenti sull'innovazione tecnologica, per poter arrivare ad avere una flotta di veicoli circolanti a bassissime emissioni, così come potrebbe essere sicuramente utile investire su motorizzazioni "pulite" come quella elettrica. Chiaramente questo tipo di interventi richiede tempo, soprattutto per quanto riguarda i motori elettrici che al momento presentano ancora diverse problematiche da risolvere, e ci dimostra come non sia possibile intervenire sugli inquinanti pensando di abbatterne le concentrazioni rapidamente.

\section{PM10 e PM2.5}
Per quanto riguarda le polveri sottili e l'andamento delle concentrazioni nel periodo di lockdown la situazione risulta essere più complessa rispetto agli ossidi di azoto.
Se da un lato, infatti, i cali di traffico e produzione industriale hanno sicuramente portato ad un calo delle emissioni, dall'altro si è visto un aumento di quelle derivanti da riscaldamenti domestici, causato da un maggior fabbisognio giornaliero delle famiglie costrette a restare più tempo nell'abitazione a causa delle misure imposte. 

%inserisci grafici PM10

Per quanto riguarda la frazione più spessa del particolato (PM10) vediamo come le concentrazioni misurate si mantengano piuttosto in linea con quanto visto negli anni precedenti, con episodi come quello di fine Marzo in cui le concentrazioni hanno fatto registrare valori ben al di sopra del limite di legge, nonostante fossero già in vigore tutte le limitazioni imposte coi vari DPCM per il contenimento dei contagi.
La serie normalizzata, invece, evidenzia un leggero calo rispetto agli anni precedenti, che però potrebbe essere collegato al leggero trend decrescente che era già stato visto sulle serie normalizzate di questo inquinante.

%inserisci grafici PM2.5

Esattamente come visto in precedenza, anche per la frazione più fine di particolato le concentrazioni si mantengono abbastanza invariate rispetto agli anni precedenti, con una piccola discrepanza sulle serie normalizzate nuovamente dovuta alla presenza di un trend leggermente calante individuato per le serie normalizzate di questo inquinante,

Per quanto riguarda il particolato, quindi, nonostante la riduzione del traffico e delle attività produttive, il maggior fabbisogno energetico delle famiglie ha fatto sì che le concentrazioni rimanessero comunque su livelli più o meno simili a quelli degli anni precedenti.
Questo ci mostra come effettivamente per trattare questi inquinanti, la cui origine è sicuramente varia e complessa, non si possano prendere provvedimenti riguardanti solamente un settore in particolare, come potrebbe essere il traffico, ma vanno pensati dei piani che riguardino tutte le fonti emissive.
La presenza di un trend negativo per tutte e due le frazioni di particolato è sicuramente un'indicazione che comunque la strada che si sta percorrendo negli ultimi anni sia quella giusta e che sia necessario proseguire investendo sempre di più su innovazione tecnologica e fonti rinnovabili.

\section{CO}
Anche per quanto riguarda il monossido di carbonio il quadro emissivo è cambiato durante i mesi dell'epidemia, con l'importante riduzione del traffico che però anche in questo caso è stata bilanciata dal maggior uso dei riscaldamenti domestici. Traffico e combustioni non industriali, infatti, sono le due categorie maggiormente responsabili delle emissioni di questo inquinante in atmosfera.

%inserisci grafici

Successivamente all'applicazione delle limitazioni tramite l'emanazione dei DPCM del mese di Marzo si nota come effettivamente le concentrazioni misurate di monossido di carbonio risultino essere leggermente minori rispetto a quelle degli anni precedenti. Tale differenza viene confermata anche dalle serie normalizzate, anche se bisogna ricordare che si tratta di differenze minime su valori già molto bassi.

Le limitazioni imposte, quindi, hanno avuto qualche effetto sulle concentrazioni di monossido di carbonio, anche se ormai la situazione per questo inquinante non risulta più essere così critica e quindi c'è anche meno interesse nella ricerca di nuove misure di contenimento, visto che quelle già presenti risultano essere più che sufficientemente efficaci.

\section{Benzene}
Il benzene è sicuramente un inquinante importante da analizzare in questo periodo, visto che spesso viene indicato come tracciante del traffico veicolare (specialmente quello motorizzato a benzina) che proprio in questi mesi ha subito importanti limitazioni e potrebbe quindi aver portato ad un calo delle concentrazioni misurate.

%inserisci grafici

Anche per questo inquinante, così collegato al traffico, notiamo uno scostamento rispetto all'andamento degli anni precedenti abbastanza comptabile con quello visto sugli ossidi di azoto.
Questo scostamento, confermato anche dalle serie normalizzate, conferma come effettivamente il traffico influenzi le concentrazioni di questo inquinante e quindi come una sua riduzione abbia conseguentemente portato ad un abbassamento delle concentrazioni.

Sebbene la situazione per quanto riguarda il benzene non desti preoccupazioni, visto che i livelli ormai sono ampiamente sotto alle soglie di legge, abbiamo visto come una riduzione del traffico abbia portato ad un ulteriore miglioramento della situazione per quanto riguarda questo inquinante, che è auspicabile venga portato avanti dall'innovazione tecnologica nel corso dei prossimi anni.

\section{SO2}
Analizziamo l'andamento delle concentrazioni di biossido di zolfo durante i mesi di lockdown, per vedere se notiamo differenze rispetto agli anni precedenti, pur ricordando che si sta trattando un inquinante le cui concentrazioni sono ormai prossime al fondo naturale.

%inserisci grafici

Si vede infatti come le concentrazioni si siano mantenute su un livello praticamente in linea con quello degli anni precedenti e come anche le serie normalizzate siano piuttosto simili, anche se per il 2020 risulta effettivamente esserci stato un leggero calo, di cui però, vista la modesta quantità, è difficile stabilire se sia dovuto alle limitazioni imposte durante il lockdown o se la sua origine sia anche dovuta ad altri fattori (come ad esempio fluttuazioni nel fondo naturale).

\section{Ozono}
L'ozono, essendo un inquinante secondario, ha origine molto varie e quindi può essere interessante valutare quali siano state le concentrazioni registrate a seguito dei provvedimenti di lockdown. 

%inserisci grafici

Vediamo come le concentrazioni per quest'anno si siano mantenute abbastanza in linea con quelle degli anni precedenti mostrando il classico trend crescente dei mesi primaverili che caratterizza questo inquinante, con la serie normalizzata che segnala un aumento, dovuto probabilmente anche alla presenza del trend positivo citato in precedenza.

Nonostante i grossi cambiamenti visti nei mesi di lockdown, quindi, l'ozono non risulta essere stato colpito da tali misure. Questo inquinante, considerate le sue complesse e varie origini, risulta quindi difficile da trattare e trovare misure per il suo contenimento è molto complesso. Abbiamo comunque visto come un blocco generale delle attività non sia stato sufficiente per migliorarne la situazione.

\section{Ammoniaca}
In ultimo andiamo a controllare l'andamento dell'ammoniaca, per la quale le serie normalizzate dei nostri modelli avevano individuato un andamento piuttosto fluttuante.

%inserisci grafici

Effettivamente controllando le concentrazioni misurate si note come tra Marzo e Maggio si sia visto un effettivo calo dei valori misurati, individuato anche dalle serie normalizzate.
Questo calo probabilmente deriva da una minor attività anche per quanto riguarda il settore agricolo, che è praticamente l'unico responsabile delle concentrazioni misurate di questo inuqinante, che è stato costretto a rallentare durante i mesi di lockdown.%rivedi/trova cit a supporto

Visto il forte collegamento tra questo inquinante e l'agricoltura qualsiasi misura di contenimento delle concentrazioni dovrà necessariamente agire su tale settore, poichè effettivamente con un calo dell'attività si è notato anche un calo nelle concentrazioni rilevate.

\section{Conclusioni}
Per quanto riguarda gli ossidi di azoto, il benzene ed il monossido di carbonio abbiamo visto come effettivamente si possa notare un calo delle concentrazioni rispetto agli anni precedenti durante i mesi di lockdown. Tale calo, visibile già dai valori misurati, è stato confermato anche dalle serie normalizzate.

Le polveri sottili, invece, essendo collegate a diverse fonti emissive, non hanno fatto registrare cali consistenti rispetto agli anni precedenti, probabilmente perchè c'è stato una compensazione tra settori che hanno ridotto le emissioni e quelli che hanno visto un aumento. Le serie normalizzate hanno mostrato una leggera diminuzione per l'anno 2020 che è però compatibile col trend decrescente individuato dai nostri modelli.

L'ammoniaca, vista la riduzione delle attività agricole, ha fatto registrare un lieve calo rispetto agli anni precedenti, mentre per l'ozono le serie normalizzate rilevano un aumento compatibile col trend crescente già individuato sulla serie.

Si è mostrato quindi come per intervenire sulle concentrazioni di inquinanti vadano previsti piani sia a lungo termine che su larga scala. La situazione per quanto riguarda la Lombardia non è sicuramente preoccupante e l'unico modo per intervenire ulteriormente sulle concentrazioni sembra proprio ridurre le emissioni, tramite il sempre maggior uso di nuove tecnologie e l'impiego di fonti di energia pulite.
Se persino un periodo come il lockdown, che si è esteso per due mesi e ha visto un'importante riduzione di tutte le attività, ha avuto effetti piuttosto marginali è chiaro come interventi sul breve termine o presi solo per particolari località possano far ben poco per migliorare la situazione.

\chapter{Conclusioni}
%TODO, IMPORTANTE!!!!

\bibliographystyle{plain}
\bibliography{Biblio}
%\addcontentsline{toc}{chapter}{Bibliografia}

\end{document}
