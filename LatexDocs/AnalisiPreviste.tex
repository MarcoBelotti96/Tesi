\documentclass{article}

\title{Specifica analisi previste per l'attività di tesi}
\author{Marco Belotti}

\begin{document}
\maketitle

In questo documento andrò ad elencare e descrivere le varie analisi che intendo affrontare nel corso della mia tesi. 
\\Inizialmente cercherò di concentrarmi sulla specifica delle varie attività, descrivendo quali confronti intendo operare e quali dati andrò ad analizzare per farlo.
\\In seconda fase sarà opportuno definire le tecniche da adottare per le analisi da svolgere, come ad esempio la normalizzazione dei dati o la ricerca della presenza di un trend in una serie.
\\Le specifiche tecniche per la parte di raccolta, elaborazione e presentazione dati saranno valutate quando sarà chiaro il quadro delle analisi da svolgere.
\newpage

\section{Introduzione}
Nel corso di questo studio intendo affrontare delle analisi sui livelli degli inquinanti e su come questi vengono influenzati da diversi fattori, facendo una ricerca su quali sono le condizioni e le misure che più influenzano i valori regisgtrati. È ormai verificato che dal 1990 in poi i livelli di quasi tutti gli inquinanti abbiano subito un forte calo (SOx -94\%; NOx -66\%; CO -
68\%), sostanzialmente dovuto a due fattori principali: l'introduzione di nuove normative atte a limitare le emissioni prodotte dalle attività umane e l'innovazione tecnologica che ha coinvolto qualsiasi settore (trasporti, industrie, produzione energetica, ecc..). Ritengo quindi che in uno studio che vuole indagare sull'andamento degli inquinanti nel corso degli anni questi due siano i fattori che devono avere maggiore considerazione. Principalmente l'idea è quella di occuparsi di tre temi: inquinanti dovuti al traffico e misure adottate, impatto dell'utilizzo di fonti energetiche rinnovabili e come si è evoluta la situazione durante l'epidemia di COVID-19.
\\Credo comunque che in prima fase sia necessario delineare un quadro della situazione, presentando dei grafici che mostrino effettivamente questo calo, andando ad analizzare tutto lo storico dei dati a disposizione. L'idea sarebbe quella di controllare ogni inquinante e produrre un grafico riportante l'andamento anno per anno dei valori registrati, rappresentadoli attraverso dei box plot, in modo che sia più visibile il range dei valori che semplicemente la media aritmetica degli stessi. È infatti importante ricordare che i livelli dei vari inquinanti sono soggetti a molte variazioni, dovute alle condizioni climatiche che possono favorirne l'accumulo o la dispersione e a molti altri fattori, ed è quindi opportuno considerare degli intervalli pittuosto che un unico valore come potrebbe essere la media aritmetica.
Inoltre potrebbe essere utile andare ad analizzare la frequenza dei superamenti delle soglie che hanno fatto registrare i diversi inquinanti, in modo da capire quali sono i più critici e che sicuramente meritano analisi più approfondite. Ad esempio gli ossidi di zolfo (SOx), che una volta erano tra gli inquinanti più preoccupanti, ormai sono stati portati a livelli trascurabili e quindi potrebbe risultare inutile basare su di essi un'analisi relativa ad un periodo recente.
\\Si potrebbe inoltre cercare di mettere in correlazione (ad esempio mediante la presentazione di scatter plot, così come usando il calcolo dell'indice di correlazione) i livelli degli inquinanti con i rilevamenti effettuati dalle stazioni meteo. Questo dovrebbe permetterci di constatare come determinate condizioni climatiche influenzino i livelli degli inquinanti e può sicuramente aiutarci nella comprensione di quali siano i fattori puù influenti. Queste constatazioni saranno inoltre molto utili nelle analisi successive, specialmente quelle che riguarderanno periodi limitati (epidemia COVID e blocchi del traffico), per permetterci di capire quanto la componente meteo possa aver influenzato il comportamento dei livelli di inquinamento.
\newpage

\section{Effetti del traffico}
Un'analisi che è sicuramente interessante affrontare è quella riguardante l'efficacia dei blocchi del traffico ed il loro impatto sui livelli degli inquinanti. 
\\Gli inquinanti maggiormente collegati al traffico sono gli ossidi di azoto (NOx), il monossido di carbonio (CO) e le polveri sottili (PM10 e PM2.5), quindi saranno da considerare stazioni e registrazioni riguardanti questi elementi.
Per capirne l'andamento è innanzitutto fondamentale verificare la presenza di un trend nel corso degli anni, in modo da poter stimare quali sarebbero i valori registrati in mancanza di interventi atti a limitarli. Una delle possibili tecniche utilizzabili è l'applicazione del test di Kendall corretto per la stagionalità, come suggerito dallo studio di ISPRA \cite{cattani2014analisi}, che permette di verificare l'esistenza di un trend eliminando la relazione che i livelli degli inquinanti hanno con i mesi dell'anno, in quanto la stagione influisce fortemente sulle concentrazioni registrate per i vari componenti.   
\\La metodologia con cui si affronterà questa ricerca sarà da approfondire e probabilmente richiederà anche una fase di normalizzazione dei dati per cercare di "pulirli" dagli effetti di fattori come il meteo o l'altezza del PBL (Primary Boundary Layer). 
\\Per affrontare questa analisi prevedo l'utilizzo di un doppio confronto: il primo andrà a verificare, su una stazione che è stata coinvolta dal blocco del traffico, se tra prima e dopo l'applicazione della restrizione si possono notare degli effetti oppure se la situazione rimane invariata.
\\Il secondo confronto da mettere in campo va a comparare i dati di una stazione coinvolta dal blocco con quelli di una che non è stata interessata, valutando se nello stesso periodo ci sono differenze di comportamento dei livelli degli inquinanti che possano suggerire un'effettiva efficacia delle misure. Ovviamente le due stazioni che saranno messe a confronto dovranno essere di due località con condizioni simili, quindi ad esempio si può considerare una zona periferica della città per confrontarla con una più centrale colpita dal blocco. 
\\Sarà poi da considerare anche le diverse tipologie di blocchi che vengono imposti, che sostanzialmente si possono raggruppare in due categorie: misure a lungo termine (Area C, Area B, Ztl, ecc..) e misure temporanee (blocchi del traffico di qualche giorno), per verificare se e in quali casi queste misure portano a dei benefici reali. 
\\L'evoluzione tecnologica nel mondo dell'auto ha comportato una serie di miglioramenti che hanno drasticamente fatto calare le emissioni prodotte dai veicoli, sia a livello di componenti che nei carburanti utilizzati. Dal 1990, infatti, inquinanti come il CO, di cui una volta il settore dei trasporti era responsabile del 65\% delle emissioni totali, hanno visto un netto calo, grazie all'applicazione di innovazioni come le marmitte catalitiche. Un altro esempio è quello degli SOx, derivati dalla combustione dello zolfo, che una volta era presente in alte concentrazioni nei carburanti, mentre ora il suo impiego è limitato per legge. Si può quindi fare un'analisi sui livelli registrati da questi due inquinanti, per vedere come l'applicazione di queste norme e l'ammodernamento della flotta di veicoli abbiano portato a un abbassamento delle concentrazioni misurate nel corso degli anni.

\subsection{Inquinanti da considerare}
Come indicato dalla relazione pubblicata da ISPRA \cite{iir2020}, gli inquinanti più rilevanti per quanto riguarda il traffico sono (dati relativi al 2018): 
\begin{itemize}
	\item NOx, con il 43\% della responsabilità sul totale delle emissioni
	\item CO, con circa il 20\% della responsabilità sul totale delle emissioni
	\item PM10, con il 12\% della responsabilità sul totale delle emissioni
	\item PM2.5, con il 10\% della responsabilità sul totale delle emissioni
\end{itemize}
Questi dati sono molto importanti, in quanto ci permettono di capire di quali livelli conviene occuparsi durante le nostre analisi, soprattutto ci indicano quali bisogna monitorare per verificare il funzionamento delle norme relative al traffico. 
\\Inoltre nello stesso rapporto sono riportati una serie di informazioni molto interessanti per quanto riguarda il traffico e la tipologia di veicoli che circolano sulle strade. Si potrà quindi pensare ad una loro integrazione nel corso delle nostre analisi.
\newpage

\section{Impatto delle fonti d'energia rinnovabili}
Cercare di analizzare l'impatto dell'utilizzo di fonti d'energia rinnovabili sulla qualità dell'aria non è sicuramente un'attività semplice, soprattutto perchè la produzione di energia pubblica influenza sui livelli degli inquinanti in modo ridotto (vedi sezione successiva).
\\Si potrebbe comunque provare a fare un'analisi legata ai livelli degli inquinanti più caratteristici di questo settore, per cercare di evidenziare delle relazioni tra i trend in calo e il sempre maggior impiego di fonti rinnovabili, soprattutto per quanto riguarda quegli inquinanti che una volta erano largamente prodotti da questo settore ed invece ora, grazie all'innovazione tecnologica, alle nuove norme e all'impiego di energie pulite, sono ampiamente calati, come ad esempio SOx, NOx e PM10.
La difficoltà di questo tipo di analisi è legata al fatto che comunque tutti gli inquinanti seguono dei trend in calo, perciò riuscire a calcolare quanto può essere stata l'influenza del maggior impiego di fonti rinnovabili non sarà sicuramente semplice. Un'aiuto può venire dai dati che Terna, la società che gestisce le reti ad alta tensione in Italia, fornisce relativi ai consumi energetici nazionali, indicando anche la ripartizione sulle varie fonti di generazione. Si potrebbe quindi provare a verificare se nel corso degli anni la quantità di energia prodotta da centrali termiche si è ridotta, per poi successivamente andare a controllare se questo calo può essere compatibile con i trend registrati dagli inquinanti, in particolar modo di quelli legati a questo tipo di produzione. In alternativa, o come analisi di supporto, è possibile andare a monitorare gli andamenti degli inquinanti, ovviamente considerando quelli caratteristici di questo settore, in relazione con le varie normative che nel corso degli anni sono state adottate per il loro contenimento. Un esempio potrebbero essere quelle norme che sono intervenute sulle concentrazioni di zolfo contenute nei combustibili, soprattutto quelli che una volta erano largamente usati come forma di riscaldamento domestico, per vedere se dell'applicazione di questi provvedimenti si possono notare dei miglioramenti concreti sui livelli registrati.
\\Un'altra analisi che si può condurre per cercare di dimostrare l'impatto dell'utilizzo di determinate fonti sulla qualità dell'aria e l'importanza delle norme adottate, è quella legata all'utilizzo di stufe e caminetti a legna come forme di riscaldamento domestico. ARPA Lombardia, nell'inventario regionale sulle emissioni degli inquinanti atmosferici\cite{inemar2017}, indica la legna come la responsabile del 50\% delle emissioni di PM10 e del 25\% delle emissioni di CO. Si potrà quindi andare a verificare le rilevazioni effettuate per questi inquinanti in località dove questa forma di riscaldamento è maggiormente utilizzata (quindi località montane e rurali), per confrontarle con quelle di una località in cui l'utilizzo della legna come forma di riscaldamento domestico non è permesso (aree urbane e paesi sotto i 300 metri di quota), così da controllare l'influenza di questo aspetto. Dovremo sicuramente tener conto che tra le due località le concentrazioni normalmente possano già differire, quindi al posto che confrontare semplicemente i valori registrati si può pensare di calcolare la variazione delle concentrazioni tra la stagione estiva e quella invernale, che dovrebbe essere più marcata nel caso in cui in inverno si utilizzi maggiormente la legna. Inoltre si potrà andare a verificare su base diurna come sono distribuite le concentrazioni e in quali momenti si hanno quelle maggiori. Nella località dove l'utilizzo di legna non è permesso dovrebbero corrispondere ai momenti di maggior traffico, mentre nell'altra potrebbero essere più alte nei momenti di maggior utilizzo dei riscaldamenti domestici (quindi sera e notte).

\subsection{Inquinanti da considerare}
Come indicato dalla relazione pubblicata da ISPRA \cite{iir2020}, gli inquinanti più rilevanti per quanto riguarda la produzione di energia sono (dati relativi al 2018): 
\begin{itemize}
	\item SOx, con l'8\% della responsabilità sul totale delle emissioni (ma nel 1990 questa percentuale era di oltre il 50\%)
	\item NOx, con circa il 4\% della responsabilità sul totale delle emissioni
	\item SOx, NOx e PM10 prodotti da questo settore negli ultimi anni hanno comunque visto un forte calo
\end{itemize}
Come già accennato l'incidenza di questo settore sul totale delle emissioni è ormai contenuta, ma non è sempre stato così storicamente. Gli inquinanti che sicuramente sarà più interessante ed utile analizzare per capire l'importanza delle norme e delle tecnologie applicate nel corso degli anni sono sicuramente i SOx.
\newpage

\section{Epidemia CoVID-19}
Partendo da quanto già approfondito da ARPA Lombardia nel suo studio \cite{arpaCovid}, sarà interessante andare ad analizzare i dati sulla qualità dell'aria durante l'epidemia di COVID-19, soprattutto una volta che avremo a disposizione il quadro completo delle rilevazioni. Bisognerà effettuare dei confronti con gli andamenti degli anni precedenti, con particolare attenzione ad analizzare gli stessi mesi dell'anno, per capire se e quanto le progressive limitazioni imposte dal Governo possano aver influito sulla qualità dell'aria.
\\Ovviamente nel corso di questa analisi, come suggerito dal dottor Carlo Bozzetti, bisognerà considerare tutti i vari fattori che possono aver influenzato sui valori misurati, in modo da avere una stima più precisa. Sicuramente sarà da considerare il trend in discesa che tutti gli inquinanti registrano, così come saranno da considerare particolari condizioni atmosferiche che possono aver portato a determinati livelli registrati. È quindi pensabile che anche per questa fase bisognerà cercare di fare un lavoro di normalizzazione e pulizia dei dati.
\\Un'altra analisi che si può fare, legata al cambiamento di abitudini della popolazione dovuto alle limitazioni imposte, è vedere come su base giornaliera possono essere cambiati i livelli di inquinamento. Tipicamente, infatti, le fasce orarie caratterizzare da maggiori livelli di traffico sono chiaramente quelle in cui si registrano le concentrazioni maggiori. In questo periodo, invece, in cui i livelli di traffico sono completamente ridotti, sarà interessante vedere se c'è stato qualche cambiamento ed eventualmente si potrà anche provare ad indagare sui risultati, per ricercare quali possono essere le cause dei valori registrati (riscaldamento, maggior consumo energetico, ecc..).
\\Proprio per questo calo del traffico sarebbe utile andare ad analizzare i dati di una stazione collocata vicino ad un'importante arteria stradale, confrontandoli magari con una stazione urbana, per vedere se si possono riscontrare differenze significative o se invece i valori registrati da entrambe tendono a seguire uno stesso trend.
\newpage

\section{Prime idee per quanto riguarda la parte di codice}
Come già previsto intendo utilizzare Python e la libreria matplotlib per tutta la parte di raccolta, elaborazione e presentazione dati. I dati forniti da ARPA sono liberamente accessibili anche attraverso l'utlizzo di API, che permettono di fare direttamente delle query sul dataset per farsi ritornare i dati desiderati. Questo potrebbe essere molto d'aiuto, in quanto dovrebbe evitarci la necessità di produrre un file CSV per contenere i dati di ogni sensore.
\\Non ho ancora esplorato a fondo le possibilità offerte dalla libreria per fare i grafici, ma credo che dei possibili risultati finali siano quelli presentati nella tesi citata\cite{scolari2017evoluzione}.
\newpage

\section{Considerazioni generali}
Di seguito una serie di considerazioni riguardanti tutte le analisi previste.
\paragraph{Normalizzazione dei dati}
Nella descrizione delle analisi previste ho menzionato il processo di normalizzazione dei dati. Bisognerà capire, innanzitutto, se vale la pena tentare di compiere questo pre-processamento dei dati, soprattutto considerando il fatto di dover ottenere comunque dei dati rappresentativi della situazione reale, senza che vengano completamente stravolti. 
\\Se verrà scelto di utilizzare questo procedimento bisognerà poi definire come attuarlo. Sicuramente le variabili da tenere in considerazione sono diverse, ma penso che le più importanti da considerare siano: precipitazioni ed altezza del PBL. Bisognerà anche capire se sono disponibili i dati relativi a quest'ultima misura (ARPA Lombardia nel suo studio\cite{arpaCovid} li utilizza) oppure se esiste la possibilità di stimarli basandosi su altri fattori. Molti spunti interessanti per quanto riguarda questo argomento sono dati dal rapporto di ISPRA del 2013\cite{cattani2014analisi}, in cui si è ricercata una tecnica di analisi dei dati che fosse di tipo quantitativo e non qualitativo.
\\In una tesi realizzata da uno studente del Politecnico \cite{scolari2017evoluzione} viene utilizzato un modo più semplice di normalizzare i dati (vedi capitolo 4), in modo da poter confrontare anche anni molto distanti e quindi con concentrazioni molto diverse. Questo metodo consiste semplicemente nel rapportare le concentrazioni rilevate di ogni anno con quelle di un anno che sarà la nostra base per i confronti, così da facilitare la comparazione dei valori registrati. 
\paragraph{Difficoltà nell'analisi dei risultati}
Nelle analisi previste sicuramente tra gli obbiettivi c'è quello di cercare delle relazioni tra misure adottate (blocchi del traffico, maggior utilizzo di energie rinnovabili) ed i livelli di inquinamento. Questo non sarà sicuramente un compito semplice, soprattutto per quanto riguarda quelle analisi che riguardano periodi brevi e quindi molto più soggetti alla variabilità dei livelli degli inquinanti. Il nostro lavoro dovrà quindi tenere conto di questo aspetto, cercando di allargare il più possibile l'insieme dei dati di confronto, in modo da "attutire" questa variabilità. Ad esempio, per quanto riguarda l'epidemia COVID, il confronto dei livelli registrati nel mese di Marzo dovrà essere almeno con i dati dello stesso mese degli ultimi anni e non sono con quelli dell'anno scorso, così da avere un miglior quadro della situazione.
\\Un altro aspetto che sarà importante ricordare in queste analisi è che tutti i livelli presentano dei trend e quindi bisognerà tenerne conto per avere una stima più precisa dei reali ribassamenti/annalzamenti registrati.
\\Inoltre, quando si trattano analisi che cercano di stimare l'influenza di determinate misure sulla qualità dell'aria, non sarà semplice fornire dei risultati certi e bisognerà essere cauti a non trarre conclusioni errate o influenzate da altri fattori. Sarà quindi opportuno che ogni attività prevista veda lo svolgimento di più analisi, che possano darci una certa confidenza sulla validità dei risultati ottenuti, senza cadere nel rischio di presupporre correlazioni basate solo su un semplice confronto.
\newpage

\bibliographystyle{ieeetr}
\bibliography{References}
\end{document}
