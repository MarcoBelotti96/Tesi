\documentclass{article}

\usepackage[utf8]{inputenc}
\usepackage[italian]{babel}
\usepackage[hyphens]{url}

\usepackage{graphicx}

\usepackage{listings}

\usepackage{biblatex}
\addbibresource{References.bib}


\usepackage{paralist}
\usepackage{layout}

\usepackage{graphicx}
\usepackage{fancyhdr}

\usepackage[a4paper,top=1cm,bottom=1cm,outer=3cm,inner=1cm,verbose,headheight=1cm,heightrounded]{geometry}
\setlength{\marginparwidth}{2.5cm} %per farci stare todonotes

\usepackage{refcheck}
\usepackage[colorinlistoftodos]{todonotes}
\usepackage[breaklinks=true]{hyperref} % interessante https://en.wikibooks.org/wiki/LaTeX/Labels_and_Cross-referencing

\title{Specifica analisi previste per l'attività di tesi}
\author{Marco Belotti}

\begin{document}
\maketitle
\tableofcontents
\listoftodos
\newpage

In questo documento andrò ad elencare e descrivere le varie analisi che intendo affrontare nel corso della mia tesi. 

\todo{atrent: ho visto che hai commentato alcuni todo ma mi sembra che tu non abbia aggiunto cose, magari quelli lasciali come nota futura, non è obbligatorio assorbirli tutti subito}

\section{Introduzione}
\label{sec:introduzione}
Nel corso di questo studio intendo affrontare delle analisi sui livelli degli inquinanti e su come questi vengono influenzati da diversi fattori, facendo una ricerca su quali sono le condizioni e le misure che più influenzano i valori regisgtrati. È ormai verificato che dal 1990 in poi i livelli di quasi tutti gli inquinanti abbiano subito un forte calo (SOx -94\%; NOx -66\%; CO -68\%)\cite{iir2020}, sostanzialmente dovuto a due fattori principali: l'introduzione di nuove normative (vedi sottosezione relativa\ref{subsec:normative})
%\todo{atrent: ref o nota sul progresso delle categorie EuroX ecc.} ho inserito i principali riferimenti per quanto riguarda le normative nella sottosezione dedicata
atte a limitare le emissioni prodotte dalle attività umane e l'innovazione tecnologica che ha coinvolto qualsiasi settore (trasporti, industrie, produzione energetica, ecc..). Ritengo quindi che in uno studio che vuole indagare sull'andamento degli inquinanti nel corso degli anni questi due siano i fattori che devono avere maggiore considerazione.

Principalmente l'idea è quella di occuparsi di tre temi: inquinanti dovuti al traffico e misure adottate, produzione energetica e fonti rinnovabili e come si è evoluta la situazione durante l'epidemia di COVID-19. Andrò quindi ad analizzare l'impatto delle misure messe in atto e dell'innovazione tecnologica riguardanti questi tre apsetti, cercando di verificare quali reali effetti si sono avuti sui livelli degli inquinanti registrati.

Credo comunque che in prima fase sia necessario delineare un quadro della situazione, presentando dei grafici che mostrino effettivamente i trend in calo delle concentrazioni, andando ad analizzare tutto lo storico dei dati a disposizione. L'idea sarebbe quella di controllare ogni inquinante e produrre un grafico riportante l'andamento anno per anno dei valori registrati, rappresentadoli attraverso dei box plot, in modo che sia più visibile il range dei valori che semplicemente la media aritmetica degli stessi. È infatti importante ricordare che i livelli dei vari inquinanti sono soggetti a molte variazioni, dovute alle condizioni climatiche che possono favorirne l'accumulo o la dispersione e a molti altri fattori, ed è quindi opportuno considerare degli intervalli pittuosto che un unico valore di riferimento come potrebbe essere la media aritmetica.
Inoltre potrebbe essere utile andare ad analizzare la frequenza dei superamenti delle soglie che hanno fatto registrare i diversi inquinanti, in modo da capire quali sono i più critici e che sicuramente meritano analisi più approfondite. Ad esempio gli ossidi di zolfo (SOx), che una volta erano tra gli inquinanti più preoccupanti, ormai sono stati portati a livelli trascurabili e quindi potrebbe risultare inutile basare su di essi un'analisi relativa ad un periodo recente.

Si potrebbe inoltre cercare di mettere in correlazione (ad esempio mediante la presentazione di scatter plot, così come usando il calcolo dell'indice di correlazione) i livelli degli inquinanti con i rilevamenti effettuati dalle stazioni meteo. Questo dovrebbe permetterci di constatare come determinate condizioni climatiche influenzino i livelli degli inquinanti e può sicuramente aiutarci nella comprensione di quali siano i fattori puù influenti. Questa parte potrebbe comunque essere integrata nel processo di normalizzazione dati, descritto più avanti (vedi sezione 'Aspetti importanti').

%\todo{atrent: da qualche parte fare anche un'analisi degli abbassamenti (normativi) delle soglie nel corso del tempo, che hanno fatto sembrare l'inquinamento in salita, riportando bib che spieghi perché lo hanno fatto (secondo me inutilmente, ma non posso dimostrarlo)}
Credo che la questione dei progressivi abbassamenti si possa integrare nella sopracitata analisi dei superamenti delle soglie. Si potrebbe controllare sia il numero dei superamenti che il rapporto tra la media dei valori registrati e il valore della soglia. Questo ci permetterebbe di vedere come le concentrazioni di alcuni inquinanti, che negli ultimi anni potrebbero aver fatto registrare un numero maggiore di superamenti (e quindi anche un rapporto più alto), in realtà siano comunque in calo, anche se sembrerebbe il contrario. Gli abbassamenti delle soglie nel tempo sono messi in atto in accordo alla politica europea di abbattimento delle emissioni di inquinanti (e gas serra), che ormai sono in atto da più di un decennio. L'obbiettivo dichiarato di queste politiche, che prevedono anche piani a lungo termine (addirittura fino al 2050), è quello di puntare sempre a una migliore qualità dell'aria che comporti sempre meno rischi per la salute umana.
%\todo{atrent: ho anche il sospetto che si stia cercando di ottenere qualcosa di non ottenibile, cioè limiti al limite (scusa il gioco di parole) del fondo naturale, se non erro in USA hanno limiti più alti, se riesci a trovare qualche ref meglio}
Come citato nella Sezione 5.1 della \href{https://ec.europa.eu/environment/air/pdf/SWD_2019_427_F1_AAQ%20Fitness%20Check.pdf}{relazione sul fitness check condotto dall'UE} sulle norme di qualità dell'aria imposte, l'obbiettivo delle norme europee è quello di mantenere le concentrazioni degli inquinanti sotto alle soglie che il WHO indica come rischiose per la salute, che in alcuni casi sono addiruttra più basse degli attuali limiti normativi comunitari. Viene però anche indicato come ci siano degli inquinanti che sistematicamente continuano a far registrare superamenti delle soglie in tutti gli stati membri dell'Unione e questo ci indica come siano ancora necessari radicali cambiamenti socio-economici per raggiungere obbiettivi che sono stati fissati dieci anni fa.
Io credo che questi continui superamenti siano semplicemente un'indicazione del fatto che il nostro stile di vita comporti certe emissioni, su cui sicuramente è importante lavorare, ma che purtroppo dobbiamo accettare e che questo comunque non significhi che l'aria sia sempre più inquinata, come invece spesso si sente dire. Questa "credenza", che viene smentita anche nella relazione, ha origine probabilmenete dalla sempre maggior attenzione della popolazione relativamente a questi temi, ma non ha nessun fondamento scientifico. Avere inquinanti che sistematicamente fanno registrare superamenti signfica che senza azioni importanti sarà difficile far ulteriormente scendere i livelli degli inquinanti, in quanto l'impegno messo in campo per limitare le emissioni negli ultimi anni è stato elevato e forse potremmo aver raggiunto una sorta di limite sotto al quale sarà difficile scendere a meno di stravolgere completamente le nostre abitudini. In tal senso mi sembra che le norme messe in campo negli anni '90 e primi 2000 siano state molto più efficaci nell'abbattimento delle concentrazioni di quelle applicate nell'ultimo periodo, facendo notare come ci si stia sicuramente avvicinando ad un limite. Si potrebbe perciò provare ad affrontare l'argomento nel corso delle nostre analisi, andando a verificare l'efficacia delle misure prese nel corso degli anni, per vedere se pian piano stiano diventando sempre meno efficaci ed applicabili, cosa che appunto potrebbe farci pensare che scendere più del livello attuale sia veramente difficile.
Per quanto riguarda il paragone coi limiti degli USA, come si può vedere dalle tabelle \href{https://en.wikipedia.org/wiki/National_Ambient_Air_Quality_Standards}{limiti americani} e \href{https://ec.europa.eu/environment/air/quality/standards.htm}{limiti europei}, pur essendo abbastanza simili, i limiti americani in alcuni casi sono più permissivi di quelli europei, in particolare per quanto riguarda PM10 e NOx. Come indicato in questo studio\cite{kuklinska2015air}, i limiti americani sono generalmente stabiliti per ridurre al minimo l'impatto degli inquinanti sulla salute della gente, mentre quelli europei sono più stringenti perchè puntano ad annullare ogni possibile effetto dell'inquinamento sulla popolazione (facendo affidamento alle raccomandazioni del WHO), ma sostanzianzialmente le politiche messe in atto da entrambi i paesi sono abbastanza simili ed i risultati ottenuti anche, quindi fatico ad immaginare di ottenere delle indicazioni utili da un confronto di questo tipo. 
\todo[color=green]{mbelotti: c'è sicuramente spazio per miglioramenti per quanto riguarda questa questione, ma al momento sono un po' a corto di idee}

\subsection{Normative}
\label{subsec:normative}
In questa sezione indicherò brevemente le normative di riferimento nel corso dei miei studi.

Per quanto riguarda le normative italiane e regionali, nella tesi\cite{scolari2017evoluzione} è sicuramente contenuto un buon riassunto di tutte le normative introdotte nel corso del tempo. Sostanzialmente, comunque, si è sempre continuato ad aumentare il numero degli inquianti considerati, abbassando progressivamente le soglie con l'obbiettivo di raggiungere standard elevati (forse anche troppo?), oltre che a limitare l'utilizzo di particolari combustibili e all'obbligo del controllo delle emissioni. 

Per quanto riguarda il traffico è sicuramente importante considerare e conoscere gli standard europei sulle emissioni inquinanti, ovvero le famose categorie EuroX che vengono applicate ai veicoli per indicare il loro livello emissivo. Attualmente la categoria in vigore è quella degli Euro6, introdotta nel 2014 e di cui tuttora non è previsto il superamento, che con le progressive revisioni è arrivata a raggiungere dei livelli emissivi davvero bassi (come, ad esempio, riportato in questo \href{https://motori.corriere.it/motori/attualita/20_aprile_27/sorpresa-diesel-non-inquina-quasi-piu-ma-restano-pregiudizi-0003501e-886a-11ea-96e3-c7b28bb4a705.shtml}{articolo del Corriere}), che rendono quindi le automobili ancora meno responsabili dell'inquinamento registrato. La tabella riportante i valori limite delle varie categorie è consultabile a \href{https://en.wikipedia.org/wiki/European_emission_standards}{questo link}. 
Altre importanti normative da considerare in questo ambito sono quelle che hanno progressivamente limitato le conecntrazioni di zolfo nei combustibili per contenere le emissioni di SOx, così come molto importante nella riduzione delle emissioni di CO è stato l'obbligo di installazione di marmitte catalitiche sugli autoveicoli. Altra importante introduzione è stata quella dei filtri antiparticolato, che, se utilizzati in condizioni ottimali, possono arrivare ad eliminare gran parte del PM dai gas di scarico. Perchè questi filtri funzionino più efficacemente possibile è però necessario che l'automobile percorra lunghe distanze, ad una velocità abbastanza sostenuta, così che si possano ottenere le alte temperature necessarie per l'eliminazione del particolato, rendendolo quindi abbastanza inutile in situazioni di stop\&go o di guida a basse velocità come invece di solito avviene in città (e questo è sicuramente un motivo per cui misure che abbassano i limiti di velocità per tentare di combattere l'inquinamento sono totalmente inutili, se non addirittura controproducenti).
 
Per quanto riguarda il settore energetico nel corso degli anni novanta sono state introdotte una serie di misure atte a limitare l'inquinamento prodotto da questo settore (ad esempio il DPR203/88), con particolare riferimento a SOx, NOx e PM10. Nel corso degli anni 2000 questi limiti sono stati poi continuamente abbassati, è stato incentivato il passaggio all'utilizzo del metano come alternativa al gasolio ed in generale si è vista l'adozione di una serie di tecniche atte ad abbattere le emissioni prodotte dalle centrali. 
Per regolare l'utilizzo di legna e pellet le prime limitazioni sono state imposte col Decreto Regionale DGR 3024/2006 e poi continuamente aggiornate fino ad arrivare all'approvazione dell'Accordo del Bacino Padano, che ha imposto limitazioni sull'uso e sull'efficienza degli impianti di riscaldamento funzionanti a biomasse.

Nel corso delle analisi legate all'epidemia COVID i procedimenti da considerare saranno tutti i decreti che progressivamente hanno bloccato le varie attività, fino ad arrivare a quello finale del 22 Marzo, in cui è stato disposto il blocco totale di qualsiasi attività ritenuta non essenziale. Visto l'evolversi della situazione credo che oltre il 4 maggio probabilmente poi si tornerà ad una situazione di pseudo-normalità e quindi probabilmente questa data potrà essere presa come termine del periodo considerato nelle nostre analisi.


\section{Effetti del traffico}
\label{sec:traffico}
Un'analisi che è sicuramente interessante affrontare è quella riguardante l'efficacia dei blocchi del traffico ed il loro impatto sui livelli degli inquinanti. 

Gli inquinanti maggiormente collegati al traffico sono gli ossidi di azoto (NOx), il monossido di carbonio (CO) e le polveri sottili (PM10 e PM2.5), quindi saranno da considerare stazioni e registrazioni riguardanti questi elementi.

Per capirne l'andamento è innanzitutto fondamentale verificare la presenza di un trend nel corso degli anni, in modo da poter stimare quali sarebbero i valori registrati in mancanza di interventi atti a limitarli. Una delle possibili tecniche utilizzabili è l'applicazione del test di Kendall corretto per la stagionalità, come suggerito dallo studio di ISPRA \cite{cattani2014analisi}, che permette di verificare l'esistenza di un trend eliminando la relazione che i livelli degli inquinanti hanno con i mesi dell'anno, in quanto la stagione influisce fortemente sulle concentrazioni registrate per i vari componenti.   

Per verificare l'utilità di questo tipo di misure prevedo l'utilizzo di un doppio confronto: il primo andrà a verificare, su una stazione che è stata coinvolta dal blocco del traffico, se tra prima e dopo l'applicazione della restrizione si possono notare degli effetti oppure se la situazione rimane invariata.
%\todo{atrent: tipo "gruppo di controllo" nel tempo e nello spazio, corretto?} 
Più che considerare un'unica stazione, forse è il caso di prendere un insieme di stazioni che siano state colpite dalla misura, per verificare come si sono comportati i livelli registrati nel corso del tempo, in modo da poter identificare meglio eventuali effetti derivati dal blocco. Se, per esempio, successivamente all'applicazione della misura tutte le stazioni mostrano dei trend in calo allora si può pensare che la stessa abbia avuto effetto, mentre se su qualche stazione si nota un calo ma su altre c'è un aumento è probabile che la restrizione non abbia avuto efficacia.

Il secondo confronto da mettere in campo va a comparare i dati di una stazione coinvolta dal blocco con quelli di una che non è stata interessata, valutando se nello stesso periodo ci sono differenze di comportamento dei livelli degli inquinanti che possano suggerire un'effettiva efficacia delle misure.
%\todo{atrent: o paesi lontani, es. milano e bormio}
Per la scelta della seconda stazione si hanno due possibilità: usare una stazione simile a quella coinvolta dal blocco, come potrebbe essere un'area più periferica della città che non è stata coinvolta dalla misura, oppure scegliere una stazione in una località lontana e con caratteristiche piuttosto differenti, come quelle di località montane o rurali. Sono tutte e due valide scelte, la prima ci permette di verificare se una misura presa può aver portato a dei reali miglioramenti oppure se applicare delle misure solo a determinate zona di una città sia inutile, mentre la seconda ci permette di fare un confronto con una situazione completamente diversa in cui è possibile verificare quale è stata l'evoluzione delle concentrazioni senza che ci siano stati interventi. Credo siano entrambe valide scelte, quindi la cosa migliore è sicuramente utilizzare entrambi i confronti nel corso delle nostre analisi.

Saranno sicuramente da considerare anche le diverse tipologie di blocchi che vengono imposti, che sostanzialmente si possono raggruppare in due categorie: misure a lungo termine (Area C, Area B, Ztl, ecc..) e misure temporanee (blocchi del traffico di qualche giorno), per verificare se e in quali casi queste misure portano a dei benefici reali. 

L'evoluzione tecnologica nel mondo dell'auto ha comportato una serie di miglioramenti che hanno drasticamente fatto calare le emissioni prodotte dai veicoli, sia a livello di componenti che nei carburanti utilizzati. Dal 1990, infatti, inquinanti come il CO, di cui una volta il settore dei trasporti era responsabile del 65\% delle emissioni totali, hanno visto un netto calo, grazie all'applicazione di innovazioni come le marmitte catalitiche. Un altro esempio è quello degli SOx, derivati dalla combustione dello zolfo, che una volta era presente in alte concentrazioni nei carburanti, mentre ora il suo impiego è limitato per legge. Si può quindi fare un'analisi sui livelli registrati da questi due inquinanti, per vedere come l'applicazione di queste norme e l'ammodernamento della flotta di veicoli abbiano portato a un abbassamento delle concentrazioni misurate nel corso degli anni. Credo che per poter provare ad ottenere dei risultati, sperando che si riesca a fare qualcosa in merito, in questo caso bisogni considerare una stazione legata al traffico, perchè dovrebbero essere quelle in cui si dovrebbero maggiormente notare gli effetti dei provvedimenti presi.

\subsection{Inquinanti da considerare}
Come indicato dall'inventario pubblicato da ISPRA \cite{iir2020}, gli inquinanti più rilevanti per quanto riguarda il traffico sono (dati relativi al 2018): 
\begin{itemize}
	\item NOx, con il 43\% della responsabilità sul totale delle emissioni
	\item CO, con circa il 20\% della responsabilità sul totale delle emissioni
	\item PM10, con il 12\% della responsabilità sul totale delle emissioni
	\item PM2.5, con il 10\% della responsabilità sul totale delle emissioni
\end{itemize}
Questi dati sono molto importanti, in quanto ci permettono di capire di quali inquinanti conviene occuparsi durante le nostre analisi, soprattutto ci indicano quali bisogna monitorare per verificare il funzionamento delle norme relative al traffico. 

Inoltre nello stesso rapporto sono riportati una serie di informazioni molto interessanti per quanto riguarda il traffico e la tipologia di veicoli che circolano sulle strade. Si potrà quindi pensare ad una loro integrazione nel corso delle nostre analisi.


\section{Produzione energetica e fonti rinnovabili}
\label{sec:energia}
Cercare di analizzare l'impatto dell'utilizzo di fonti d'energia rinnovabili sulla qualità dell'aria non è sicuramente un'attività semplice, soprattutto perché la produzione di energia pubblica influenza sui livelli degli inquinanti in modo ridotto (vedi sezione successiva).

Intendo comunque provare a fare un'analisi legata ai livelli degli inquinanti più caratteristici di questo settore, per cercare di evidenziare delle relazioni tra i trend in calo e il sempre maggior impiego di fonti rinnovabili, soprattutto per quanto riguarda quegli inquinanti che una volta erano largamente prodotti da questo settore ed invece ora, grazie all'innovazione tecnologica, alle nuove norme e all'impiego di energie pulite, sono ampiamente calati, come ad esempio SOx, NOx e PM10.
La difficoltà di questo tipo di analisi è legata al fatto che comunque tutti gli inquinanti seguono dei trend in calo, perciò riuscire a calcolare quanto può essere stata l'influenza del maggior impiego di fonti rinnovabili non sarà sicuramente semplice. Un'aiuto può venire dai dati che Terna, la società che gestisce le reti ad alta tensione in Italia, fornisce relativi ai consumi energetici nazionali, indicando anche la ripartizione sulle varie fonti di generazione. Si potrebbe quindi provare a verificare se nel corso degli anni la quantità di energia prodotta da centrali termiche si è ridotta, per poi successivamente andare a controllare se questo calo può essere compatibile con i trend registrati dagli inquinanti, in particolar modo di quelli legati a questo tipo di produzione. In alternativa, o come analisi di supporto, è possibile andare a monitorare gli andamenti degli inquinanti, ovviamente considerando quelli caratteristici di questo settore, in relazione con le varie normative che nel corso degli anni sono state adottate per il loro contenimento. Un esempio potrebbero essere quelle norme che sono intervenute sulle concentrazioni di zolfo contenute nei combustibili, soprattutto quelli che una volta erano largamente usati per il riscaldamento domestico e per la produzione enrgetica, per controllare se dall'applicazione di questi provvedimenti si possono notare dei miglioramenti concreti sui livelli registrati. 

Un'altra analisi che si può condurre per cercare di dimostrare l'impatto dell'utilizzo di determinate fonti sulla qualità dell'aria e l'importanza delle norme adottate, è quella legata all'utilizzo di stufe e caminetti a legna come forme di riscaldamento domestico. ARPA Lombardia, nell'inventario regionale sulle emissioni degli inquinanti atmosferici\cite{inemar2017}, indica la legna come la responsabile del 50\% delle emissioni di PM10 e del 25\% delle emissioni di CO. Si potrà quindi andare a verificare le rilevazioni effettuate per questi inquinanti in località dove questa forma di riscaldamento è maggiormente utilizzata (quindi località montane e rurali), per confrontarle con quelle di una località in cui l'utilizzo della legna come forma di riscaldamento domestico viene limitato con obblighi sull'efficienza dell'impianto e sulle emissioni prodotte (aree urbane e paesi sotto i 300 metri di quota), concentrandosi sulle rilevazioni fatte in periodo invernale. Dovremo sicuramente tener conto che tra le due località le concentrazioni normalmente possano già differire, quindi al posto che confrontare semplicemente i valori registrati si può pensare di calcolare la variazione delle concentrazioni tra la stagione estiva e quella invernale, che dovrebbe essere più marcata nel caso in cui in inverno si utilizzi maggiormente la legna. Inoltre si potrà andare a verificare su base diurna come sono distribuite le concentrazioni e in quali momenti si hanno quelle maggiori. Nella località dove l'utilizzo di legna non è permesso dovrebbero corrispondere ai momenti di maggior traffico, mentre nell'altra potrebbero essere più alte nei momenti di maggior utilizzo dei riscaldamenti domestici (quindi sera e notte).

\subsection{Inquinanti da considerare}
Come indicato dall'inventario pubblicato da ISPRA \cite{iir2020}, gli inquinanti più rilevanti per quanto riguarda la produzione di energia sono (dati relativi al 2018): 
\begin{itemize}
	\item SOx, con l'8\% della responsabilità sul totale delle emissioni (ma nel 1990 questa percentuale era di oltre il 50\%)
	\item NOx, con circa il 4\% della responsabilità sul totale delle emissioni
	\item SOx, NOx e PM10 prodotti da questo settore negli ultimi anni hanno comunque visto un forte calo
\end{itemize}
Come già accennato l'incidenza di questo settore sul totale delle emissioni è ormai contenuta, ma non è sempre stato così storicamente. Gli inquinanti che sicuramente sarà più interessante ed utile analizzare per capire l'importanza delle norme e delle tecnologie applicate nel corso degli anni sono sicuramente gli SOx.


\section{Epidemia CoVID-19}
\label{sec:covid}
Partendo da quanto già approfondito da ARPA Lombardia nel suo studio \cite{arpaCovid}, sarà interessante andare ad analizzare i dati sulla qualità dell'aria durante l'epidemia di COVID-19, soprattutto una volta che avremo a disposizione il quadro completo delle rilevazioni. Bisognerà effettuare dei confronti con gli andamenti degli anni precedenti, con particolare attenzione ad analizzare gli stessi mesi dell'anno, per capire se e quanto le progressive limitazioni imposte dal Governo possano aver influito sulla qualità dell'aria.

Ovviamente nel corso di questa analisi, come suggerito dal dottor Carlo Bozzetti, bisognerà considerare tutti i vari fattori che possono aver influenzato sui valori misurati, in modo da avere una stima più precisa. Sicuramente sarà da considerare il trend in discesa che tutti gli inquinanti registrano, così come saranno da considerare particolari condizioni atmosferiche che possono aver portato a determinati livelli registrati. È quindi pensabile che anche per questa fase bisognerà cercare di fare un lavoro di normalizzazione e pulizia dei dati.

Un'altra analisi che si può fare, legata al cambiamento di abitudini della popolazione dovuto alle limitazioni imposte, è vedere come su base giornaliera possono essere cambiati i livelli di inquinamento. Tipicamente, infatti, le fasce orarie caratterizzare da maggiori livelli di traffico sono chiaramente quelle in cui si registrano le concentrazioni maggiori. In questo periodo, invece, in cui i livelli di traffico sono completamente ridotti, sarà interessante vedere se c'è stato qualche cambiamento ed eventualmente si potrà anche provare ad indagare sui risultati, per ricercare quali possono essere le cause dei valori registrati (riscaldamento, maggior consumo energetico, ecc..).

Proprio per questo calo del traffico sarebbe utile andare ad analizzare i dati di una stazione collocata vicino ad un'importante arteria stradale, confrontandoli magari con una stazione urbana, per vedere se si possono riscontrare differenze significative o se invece i valori registrati da entrambe tendono a seguire uno stesso trend.

%\todo{atrent: si potrà tentare di stimare il famoso "fondo naturale"? forse difficile perché in realtà non tutte le attività si sono fermate}
Sarebbe sicuramente interessante provare a stimare il "fondo naturale" e credo che questo potrebbe essere il periodo migliore per farlo, visto il forte calo della attività umane. Purtroppo però è anche vero che non tutte le attività si sono fermate, così come gli inquinanti derivati dai riscaldamenti domestici e dalla produzione energetica non dovrebbero essere calati così drasticamente. Si potrebbe comunque pensare di indagare su alcuni inquinanti specifici derivanti principalmente da traffico e produzione industriale, che potrebbero essere quelli maggiormente colpiti da questo periodo di lockdown, per vedere se si può condurre qualche analisi in merito a questo aspetto. Sicuramente quando si avrà a disposizione tutto il quadro della situazione sarà più facile capire su quali inquinanti può valer la pena di indagare, che saranno quelli che avranno fatto registrare i maggiori cali.
A proposito della questione sarebbe interessante capire se ARPA ha iniziato a scorporare il valore del fondo naturale dai valori che poi pubblica. In rete non ho trovato particolari indicazioni in tal senso, ma è un aspetto che andrà certamente considerato.

Saranno sicuramente da tenere in considerazione i dati rilasciati da \href{https://www.apple.com/covid19/mobility}{Apple} e \href{https://www.google.com/covid19/mobility/}{Google} relativi agli spostamenti registrati nel corso dell'epidemia, perchè ci forniscono indicazioni importanti sullo scenario che vogliamo andare ad analizzare. Ad esempio dai dati si può vedere come il traffico si sia drasticamente ridotto (-77\%) a partire dal mese di marzo, quindi si potrebbe pensare che le stazioni situate vicino ad importanti arterie stradali in questo periodo abbiano registrato il fondo naturale, o comunque un valore abbastanza vicino.  


\section{Prime idee per quanto riguarda la parte di codice}
Come già previsto intendo utilizzare Python e la libreria matplotlib per tutta la parte di raccolta, elaborazione e presentazione dati. I dati forniti da ARPA sono liberamente accessibili anche attraverso l'utlizzo di API, che permettono di fare direttamente delle query sul dataset per farsi ritornare i dati desiderati. Questo potrebbe essere molto d'aiuto, in quanto dovrebbe evitarci la necessità di produrre un file CSV per contenere i dati di ogni sensore, alleggerendo così il carico elaborativo dell'applicazione.


\section{Aspetti importanti}
Di seguito una serie di considerazioni riguardanti tutte le analisi previste.
\subsection{Normalizzazione dei dati}
Come già evidenziato da diversi studi, ad esempio \cite{hoogerbrugge2010trends} e \cite{grange2019using}, la normalizzazione dei dati sull'inquinamento relativamente alle condizioni metereologiche è sicuramente una tencica molto importante, che ci permette di indagare con più precisione sia sui trend fatti registrare dalle concentrazioni che sull'efficacia di misure prese, poichè ci consente di eliminare la variabilità causata da tali elementi. Come è ben noto, infatti, pioggia, vento, umidità, pressione atmosferica e temperatura influiscono tutti sui livelli di tutti gli inquinanti, arrivando a stravolgerne completamente i valori, quindi sarebbe sicuramente un vantaggio poter tener conto di questa loro influenza nell'analisi dei valori registrati.

La tecnica che voglio applicare per raggiungere questo obbiettivo è l'algoritmo di machine learning Random Forest, un algoritmo di machine learning molto famoso e considerato valido per tutti i problemi in cui si vuole ottenere un modello utile per generare delle previsioni di una quantità di interesse a partire da un set di osservazioni di un numero di variabili dette predittori. Nel nostro caso la quantità che andremmo a predirre sarebbe il livello dell'inquinante ed i predittori sarebbero le diverse variabili metereologiche (come pioggia, direzione e velocità del vento, temperatura, ecc..) e variabili temporali (quindi la data, il giorno della settimana, il mese, ecc..). Questa tecnica è la stessa applicata nei due studi \cite{grange2019using} \cite{grange2018random}, che offrono diversi spunti interessanti sia sui procedimenti operativi nell'applicazione di questa tecnica che per quanto riguarda il suo utilizzo nelle analisi che si intendono poi compiere sui livelli degli inquinanti.

Alla base di questa tecnica ci sono gli alberi di decisione, il cui funzionamento è molto semplice. Partendo dal set di osservazioni a disposizione e utilizzando diversi criteri statistici per la generazione dei nodi dell'albero in modo che esso sia il più accurato possibile, si riesce ad ottenere un albero binario i cui nodi interni sono condizioni e le foglie invece sono il risultato della predizione fatta dell'albero di decisione. In pratica, per ogni previsione che si vuole fare per una serie di valori osservati delle variabili predittrici, si parte dalle radice, in cui è contenuta una condizione da verificare su tali valori. In base al risultato di essa si procederà poi nella metà destra o nella metà sinistra dell'albero, in cui incontreremo una nuova condizione e perciò ripeteremo ricorsivamente questi passaggi sino ad arrivare ad una foglia, che conterrà il valore della predizione fatta per i valori dei predittori dati in input.
Gli alberi di decisione hanno però un problema detto overfitting, ovvero tendono ad adattarsi bene alle serie di dati che vengono utilizzati per crearli, ma poi quando devono generare previsioni partendo da dati in input mai visti risultano poco precisi, compromettendone quindi la possibilità di utilizzarli per avere previsioni sufficientemente precise. Random forest risolve questo problema creando un numero molto grande di alberi di decisione, ognuno di essi generato a partire da un sottoinsieme diverso dei dati da cui si vuole creare il modello (detto training set). Ognuno di questi alberi viene poi usato per fare una sua previsione ed i risultati ottenuti sono aggregati, ad esempio facendo la media aritmetica, per ottenere la previsione finale del modello. È dimostrato statisticamente che questo procedimento riesce a ridurre la varianza del modello ottenuto, risolvendo quindi il problema dell'overfitting e permettendoci di usare gli alberi di decisione per ottenere predizioni accurate.
Questa tecnica ha il vantaggio di non fare assunzioni sul tipo di relazione che c'è tra le variabili predittrice e la quantità da predirre, permette che i valori delle variabili di input possano essere in relazione e scarta automaticamente le variabili che non hanno importanza nella determinazione del risultato (ovvero non considera quelle variabili che non influiscono sul valore da predirre). Questi tre punti sono i limiti della regressione lineare, la tecnica che inizialmente intendevo applicare e che poi ho deciso di abbandonare in favore di random forest, e avrebbero sicuramente richiesto uno sforzo decisamente maggiore per riuscire ad ottenere un risultato valido, e quindi accurato nel fare previsioni, come quello ottenibile con random forest. Questo non significa che sia una tecnica da applicare senza porsi nessun quesito, e infatti in questi giorni ho studiato parecchio l'argomento per capire bene come vada applicata e quali siano le considerazioni da fare durante il suo utilizzo.
Un altro vantaggio di random forset è la velocità, sia per quanto riguarda la generazione del modello che quando si vogliono ottenere delle previsioni. Inoltre è una tecnica già implementata in diversi linguaggi, compreso Python, quindi il suo utilizzo risulterebbe ulteriormente semplificato, permettendoci di concentrarci sui risultati ottenuti senza doverci preoccupare di implementare l'algoritmo, attività che sarebbe molto impegnativa e dispendiosa.

A livello pratico, quindi, ecco un sommario delle attività che occorre prevedere.
\begin{itemize}
\item \textbf{Recupero dei dati e creazione del modello}
Bisognerà essere in grado di recuperare i dati delle serie storiche relative sia agli inquinanti di interesse per una determinata stazione (o un insieme di stazioni) ed aggregarla con quelle delle rilevazioni meteo. Queste serie storiche saranno poi divise in due insiemi: il primo, detto training set, sarà usato per generare il modello utilizzando una tecnica detta bootstrap, il secondo, detto testing set, sarà invece utilizzato per verificare l'accuratezza delle predizioni fatte dal modello ottenuto a partire dal training set. Questo ci permetterà quindi di verificare di aver ottenuto un modello abbastanza valido per essere usato nella normalizzazione delle concentrazioni metereologiche.
La tecnica bootstrap riguarda invece la parte di generazione dei molteplici alberi di decisione, ognuno generato a partire da un sottoinsieme diverso del training set, come già accennato in precedenza.

\item \textbf{Tuning degli iperparametri}
La tecnica random forest ha una serie di iperparametri, che sono, ad esempio: il numero di alberi da generare nella foresta, l'altezza massima che questi alberi possono avere, il numero di variabili che si può utilizzare per creare la condizione da mettere in un nodo, ecc..
Modificando questi parametri è possibile aumentare la precisione del modello, quindi una volta che avremo ottenuto un primo modello sarà sicuramente utile esplorare i possibili miglioramenti ottenibili tramite questa operazione.

\item \textbf{Normalizzazione dei dati}
Per concludere il discorso, quindi, dovremo applicare questo modello per ottenere il valore dell'inquinante normalizzato dalle condizioni metereologiche. Visto che il modello ottenuto da random forest sarà in grado di fare previsioni precise sul livello di un inquinante basandosi sulle condizioni metereologiche registrate in quel momento, ma non è in grado di dirci come ogni singolo aspetto può incidere sul risultato finale in modo preciso, per ottenere il valore normalizzato procederemo effettuando più previsioni per tale valore. Ognuna di queste previsoni sarà effettuata considerando condizioni metereologiche estratte a caso da quelle registrate storicamente ed alla fine saranno tutte aggregate ottenendo la media, che sarà il nostro valore normalizzato. Questo procedimento si basa quindi sull'idea di ottenere un valore normalizzato poichè per ogni momento per cui si fa una previsione si considera una media di quelli che sarebbero stati i valori registrati se ci fossero state le diverse condizioni metereologiche, ottenendo quindi alla fine un valore basato su condizioni "medie". 
Questi valori normalizzati che sono ottenuti sono poi utilizzabili per fare tutte le successive analisi sfruttando tutti i vantaggi elencati dai diversi studi, come \cite{hoogerbrugge2010trends}, quali ad esempio la miglior possibilità di verificare l'efficacia delle misure prese.
\end{itemize}

Ovviamente saranno da definire le variabili che si vogliono usare come predittori per il modello, comunque un insieme di partenza potrebbe essere: unix date (ms a partire dal 1-1-1970) così da poter avere un riferimento temporale che permetta di considerare i trend storici degli inquinanti (come indicato nei diversi studi citati è utile includerla), temperatura, umidità, direzione e velocità del vento, pressione atmosferica e pioggia.
Nel corso delle analisi sarà poi eventualmente possibile fare considerazioni sull'importanza di determinate variabili e quali sia più conveniente utilizzare, quindi l'insieme finale sarà ben definibile sono una volta che potremo fare un po' di esperimenti coi modelli ottenuti a partire dai dati, vedendo se aggiungere o eliminare qualche variabile possa essere d'aiuto per aumentare la precisione delle previsioni fatte.

Questa tecnica sarà sicuramente applicabile nel corso di tutte le nostre analisi, in quanto ci permette di eliminare molta della variabilità delle concentrazioni causata dalle condizioni metereologiche. Sicuramente sarà molto interessante la sua applicazione nello studio dei blocchi del traffico, visto che dovrebbe permetterci di dimostrare ancora meglio come determinate misure siano totalmente inutili sui livelli registrati. Sarà poi molto interessante andare ad applicarla anche ai dati del periodo riguardante l'epidemia di COVID-19, per verificare come si è realmente evoluta la situazione e come si sono comportate le concentrazioni nel corso dei mesi di marzo e aprile. Non so se sarà applicabile anche per stimare il fondo naturale, ma se in questo periodo noteremo un forte ribasso anche nei valori normalizzati allora potremmo forse pensare che i valori effettivamente registrati in questo periodo siano appunto quelli causati dal fondo.

Un altro vantaggio che ci porterebbe la normalizzazione dei dati sarebbe relativo allo studio dei trend nelle concentrazione registrate, su cui spunti importanti sono dati dai rapporti di ISPRA\cite{cattani2018analisi}\cite{cattani2014analisi}, in cui si è ricercata una tecnica di analisi dei dati che permetta di fare delle analisi di tipo quantitativo sugli andamenti degli inquinanti. Come già accennato la tecnica applicata è il test di Kendall corretto per la stagionalità, il cui utilizzo può essere valido anche per il mio studio quando ci sarà la necessità di quantificare il calo registrato dalle concentrazioni. Stavo pensando ad una sua possibile applicazione nella verifica di efficacia dei blocchi del traffico, ma dovrei pensarci meglio ed approfondire alcuni dettagli per verificarne la fattibilità. 

Credo che nei prossimi giorni inizierò a sperimentare un po' con il codice per quanto riguarda questa parte, così una volta che avremo a disposizione questa funzione ci si potrà concentrare bene sulle analisi previste per i diversi settori.

\subsection{Difficoltà nell'analisi dei risultati}
Nelle analisi previste sicuramente tra gli obbiettivi c'è quello di cercare delle relazioni tra misure adottate (blocchi del traffico, maggior utilizzo di energie rinnovabili) ed i livelli di inquinamento. Questo non sarà sicuramente un compito semplice, soprattutto per quanto riguarda quelle analisi che riguardano periodi brevi e quindi molto più soggetti alla variabilità dei livelli degli inquinanti. Il nostro lavoro dovrà quindi tenere conto di questo aspetto, cercando di allargare il più possibile l'insieme dei dati di confronto, in modo da "attutire" questa variabilità. Ad esempio, per quanto riguarda l'epidemia COVID, il confronto dei livelli registrati nel mese di Marzo dovrà essere almeno con i dati dello stesso mese degli ultimi anni e non sono con quelli dell'anno scorso, così da avere un miglior quadro della situazione.

Un altro aspetto che sarà importante ricordare in queste analisi è che tutti i livelli presentano dei trend e quindi bisognerà tenerne conto per avere una stima più precisa dei reali ribassamenti/innalzamenti registrati.

Inoltre, quando si trattano analisi che cercano di stimare l'influenza di determinate misure sulla qualità dell'aria, non sarà semplice fornire dei risultati certi e bisognerà essere cauti a non trarre conclusioni errate o influenzate da altri fattori. Sarà quindi opportuno che ogni attività prevista veda lo svolgimento di più analisi, che possano darci una certa confidenza sulla validità dei risultati ottenuti, senza cadere nel rischio di presupporre correlazioni basate solo su un semplice confronto.


\section{Sommario attività}
Qui di seguito andrò ad elencare quindi le varie attività che prevedo di affrontare nel corso del mio studio.

Il primo passaggio da affrontare è sicuramente quello relativo alla normalizzazione dei dati. Sarà da mettere bene a punto una tecnica precisa, che come già accennato sarà basata sulla regressione lineare, in modo da avere poi un modello di normalizzazione valido e applicabile in tutte le altre analisi. Siccome questa tecnica richiede l'analisi delle registrazioni effettuate per poter stabilire quali relazioni legano le rilevazioni meteo con i livelli degli inquinanti, sarà sicuramente da affrontare anche il tema del recupero dei dati dai datasets, andando a creare delle funzioni che saranno poi usate anche nel resto del nostro studio.%\todo{atrent: gli url dei dataset te li avevo dati?} si si, li ho già a disposizione

Una volta che avremo un modello soddisfacente per la normalizzazione dei dati potremo passare ad affrontare le analisi descritte nella sezione 'Introduzione'\ref{sec:introduzione}. Non sono analisi complicate e non credo che richiederanno troppo sforzo, ma ci serviranno per avere un quadro generale della situazione prima di affrontare gli altri temi. Inoltre sarà possibile vedere come l'applicazione della normalizzazione andrà a variare i valori registrati, permettendoci di verificare come cambia il quadro della situazione dopo l'applicazione di questa tecnica.

Successivamente intendo affrontare le analisi descritte nella sezione relativa al traffico\ref{sec:traffico}. I confronti da affrontare saranno quelli specificati, eventualmente da supportare con ulteriori studi se li riterremo necessari. In questa fase sarà necessario implementare diverse funzionalità per il recupero e l'elaborazione dei dati, che, se ben sviluppate, saranno poi probabilmente utilizzabili anche per tutto il resto dell studio, andando solo a modificare le parti necessarie.

Andrà poi affrontato il settore della produzione energetica, andando a svolgere le analisi previste nella relativa sezione\ref{sec:energia}.

Come ultima analisi, sperando di avere un quadro completo della situazione, si potrà andare ad analizzare l'evoluzione degli inquinanti nel periodo dell'epidemia di COVID-19. Anche per questa fase le analisi previste sono quelle già descritte nella sezione relativa\ref{sec:covid}, ma potrebbero essere espanse o modificate in base all'evoluzione della situazione. ARPA Lombardia, dopo aver pubblicato la prima analisi\cite{arpaCovid}, aveva promesso di riaffrontare la questione una volta che ci sarebbe stato a disposizione il quadro completo dell'evoluzione, quindi se dovessero uscire nuovi studi saranno sicuramente da considerare e valutare se eventualmente sarà necessario apportare qualche modifica agli studi previsti. 
 

%\bibliographystyle{ieeetr}
%\bibliography{References}
\label{bib-begin}
\printbibliography
\end{document}
