\documentclass{article}

\title{Specifica analisi previste per l'attività di tesi}
\author{Marco Belotti}

\begin{document}
\maketitle

In questo documento andrò ad elencare e descrivere le varie analisi che intendo affrontare nel corso della mia tesi. 
\\Inizialmente cercherò di concentrarmi sulla specifica delle varie attività, descrivendo quali confronti intendo operare e quali dati andrò ad analizzare per farlo.
\\In seconda fase sarà opportuno definire le tecniche da adottare per le analisi da svolgere, come ad esempio la normalizzazione dei dati o la ricerca della presenza di un trend in una serie.
\\Le specifiche tecniche per la parte di raccolta, elaborazione e presentazione dati saranno valutate quando sarà chiaro il quadro delle analisi da svolgere.
\newpage

\section{Introduzione}
È ormai dimostrato che tutti i livelli degli inquinanti monitorati dalle stazioni negli ultimi 30 anni hanno un trend in calo, determinato dall'avanzamento tecnologico, sia a livello industrile che per quanto riguarda i veicoli, e dal sempre maggior utilizzo di fonti di energia rinnovabili.
\\Ritengo comunque che in prima fase sia necessario delineare un quadro della situazione, presentando dei grafici che mostrino effettivamente questo calo, andando ad analizzare tutto lo storico dei dati a disposizione. L'idea era quella di controllare ogni inquinante e produrre un grafico riportante l'andamento anno per anno dei valori registrati, rappresentadoli attraverso dei box plot, in modo che sia più visibile il range dei valori che semplicemente la media aritmetica degli stessi. È infatti importante ricordare che i livelli dei vari inquinanti sono soggetti a molte variazioni, dovute alle condizioni climatiche che possono favorirne l'accumulo o la dispersione e a molti altri fattori, ed è quindi opportuno considerare degli intervalli pittuosto che un unico valore come potrebbe essere la media aritmetica.
\\Si potrebbe inoltre cercare di mettere in correlazione (ad esempio mediante la presentazione di scatter plot, così come usando il calcolo dell'indice di correlazione) i livelli degli inquinanti con i rilevamenti effettuati dalle stazioni meteo. Questo dovrebbe permetterci di constatare come determinate condizioni climatiche influenzino i livelli degli inquinanti e può sicuramente aiutarci nella comprensione di quali siano i fattori puù influenti. Queste constatazioni saranno inoltre molto utili nelle analisi successive, specialmente quelle che riguarderanno periodi limitati (epidemia COVID e blocchi del traffico), per permetterci di capire quanto la componente meteo possa aver influenzato il comportamento dei livelli di inquinamento.
\\I livelli di molti inquinanti sono tra loro collegati, in quanto alcuni sono il prodotto delle trasformazioni subite da altri (ad esempio il monossido di azoto in atmosfera si ossida e si trasforma in biossido di azoto). Si potrebbe quindi provare ad evidenziare questa relazione a livello di grafici, dimostrando come determinati livelli di un certo inquinante possano portare a registrazioni di valori compatibili anche dei sui successori.
\newpage

\section{Efficacia dei blocchi del traffico}
Un'analisi che è sicuramente interessante affrontare è quella riguardante l'efficacia dei blocchi del traffico ed il loro impatto sui livelli degli inquinanti. 
\\Gli inquinanti maggiormente collegati al traffico sono gli ossidi di azoto (NOx) e le polveri sottili (PM10 e PM2.5), quindi saranno da considerare stazioni e registrazioni riguardanti questi elementi.
Per capirne l'andamento è innanzitutto fondamentale verificare la presenza di un trend nel corso degli anni, in modo da poter stimare quali sarebbero i valori registrati in mancanza di interventi atti a limitarli. 
\\La metodologia con qui si affronterà questa ricerca sarà da approfondire e probabilmente richiederà anche una fase di normalizzazione dei dati per cercare di "pulirli" dagli effetti di fattori come il meteo o l'altezza del PBL (Primary Boundary Layer). In ogni caso sarà comunque un'operazione da applicare praticamente a tutte le analisi che si andranno ad affrontare, quindi sarà importante adottare tecniche corrette. Una di queste queste può essere l'applicazione del test di Kendall corretto per la stagionalità, come suggerito dallo studio di ISPRA \cite{cattani2014analisi}, che permette di verificare l'esistenza di un trend eliminando la relazione che i livelli degli inquinanti hanno con i mesi dell'anno, in quanto la stagione influisce fortemente sulle concentrazioni registrate per i vari componenti.  
\\Per affrontare questa analisi prevedo l'utilizzo di un doppio confronto: il primo andrà a verificare, su una stazione che è stata coinvolta dal blocco del traffico, se tra prima e dopo l'applicazione della restrizione si possono notare degli effetti oppure se la situazione rimane invariata. Per questo tipo di confronto forse è più opportuno andare a considerare quelle misure che si protraggono nel tempo per un intervallo ampio (ad esempio l'area C), poichè avere a disposizione dei periodi più lunghi ci permette di verificare l'effettiva efficacia in modo migliore e quindi anche di poter trarre delle conclusioni più precise.
\\Il secondo confronto da mettere in campo va a comparare i dati di una stazione coinvolta dal blocco con quelli di una che non è stata interessata, valutando se nello stesso periodo ci sono differenze di comportamento dei livelli degli inquinanti che possano suggerire un'effettiva efficacia delle misure. Ovviamente le due stazioni che saranno messe a confronto dovranno essere di due località con condizioni simili, quindi ad esempio si può considerare una zona periferica della città per confrontarla con una più centrale colpita dal blocco. Con questo confronto si potranno anche studiare quelle misure temporanee, come i blocchi del traffico, per verificare se hanno un'utilità effettiva.
\\Se abbiamo a disposizione dei dati sulla tipologia dei veicoli nel corso degli anni possono magari essere utilizzati per fare delle verifiche su determinati livelli che dovrebbero corrispondere alle tipologie di motori utilizzati (ad esempio il benzene, che è prodotto maggiormente dai motori a benzina), per vedere se è possibile notare dei miglioramenti connessi all'ammodernamento del parco auto nazionale.
\newpage

\section{Impatto delle fonti d'energia rinnovabili}
Sicuramente tra i tanti fattori che hanno portato alla diminuzione dei livelli di inquinamento negli ultimi anni c'è il sempre maggior utilizzo delle fonti d'energia rinnovabili. 
\\Il GSE (Gestore Servizi Energetici) fornisce dei rapporti riguardanti l'utilizzo di fonti d'energia rinnovabili nel nostro paese nel corso degli anni. Si possono usare questi dati per andare a verificare se il sempre maggior impiego di energie pulite corrisponda al calo di determinati livelli di inquinanti, quali ad esempio il monossido di carobonio, che sono prodotti dalle centrali termiche. Ovviamente anche in questo caso sarà opportuno considerare la possibilità dell'esistenza di un trend e quindi bisognerà cercare di capire quanto realmente influisca l'aspetto delle energie pulite sui cali regitrati. Inoltre Terna, il gestore delle linee elettriche ad alta tensione in Italia, attraverso il suo sito fornisce i dati sui consumi registrati in Italia. Anche questi dati possono essere usati in supporto all'analisi discussa per verificare, mediante indici di correlazione e grafici adeguati, se e quanto, nei momenti di maggior consumo energetico, i livelli di alcuni inquinanti crescono. Questo sarebbe infatti un primo modo per andare ad analizzare quanto pesi l'impatto della produzione energetica sui livelli registrati, prima di procedere con l'analisi sopracitata.
\\Un'altra analisi che si può condurre per quanto riguarda questo aspetto è quella legata all'utilizzo di stufe e caminetti a legna come forme di riscaldamento, in quanto forti produttori di monossido di carbonio. Si potrà quindi andare a controllare le rilevazioni effettuate in località dove questa forma di riscaldamento è maggiormente utilizza (quindi località montane e rurali), considerando soprattutto quelle effettuate durante la stagione invernale. In questo modo dovremmo quindi essere in grado di capire se e quanto effettivamente questa forma di riscaldamento influisca sui livelli di monossido di carobnio. Un ulteriore controllo potrebbe essere fatto su base giornaliera, per verificare se le fasce orarie con i livelli maggiori possono corrispondere a quelle in cui teoricamente l'utilizzo di stufe e caminetti dovrebbe essere maggiore, quindi soprattutto nella fascia serale della giornata, visto che è quella in cui la gente è maggiormente nelle proprie abitazioni ed i livelli di traffico sono minori. 
\\Un'altra possibilità sarebbe quella di confrontare i dati della località scelta con quelli di una compatibile (quindi che gode di condizioni metereologiche simili) in cui però non è permesso l'utilizzo della legna come forma di riscaldamento. Anche questo potrebbe essere un possibile modo per cercare di evidenziare l'influenza di questo aspetto sui livelli di monossido di carbonio registrati.
\newpage

\section{Epidemia CoVID-19}
Partendo da quanto già approfondito da ARPA Lombardia nel suo studio \cite{arpaCovid}, sarà interessante andare ad analizzare i dati sulla qualità dell'aria durante l'epidemia di COVID-19, soprattutto una volta che avremo a disposizione il quadro completo delle rilevazioni. Bisognerà effettuare dei confronti con gli andamenti degli anni precedenti, con particolare attenzione ad analizzare gli stessi mesi dell'anno, per capire se e quanto le progressive limitazioni imposte dal Governo possano aver influito sulla qualità dell'aria.
\\Ovviamente nel corso di questa analisi, come suggerito dal dottor Carlo Bozzetti, bisognerà considerare tutti i vari fattori che possono aver influenzato sui valori misurati, in modo da avere una stima più precisa. Sicuramente sarà da considerare il trend in discesa che tutti gli inquinanti registrano, così come saranno da considerare particolari condizioni atmosferiche che possono aver portato a determinati livelli registrati. È quindi pensabile che anche per questa fase bisognerà cercare di fare un lavoro di normalizzazione e pulizia dei dati.
\\Un'altra analisi che si può fare, legata al cambiamento di abitudini della popolazione dovuto alle limitazioni imposte, è vedere come su base giornaliera possono essere cambiati i livelli di inquinamento. Tipicamente, infatti, le fasce orarie caratterizzare da maggiori livelli di traffico sono chiaramente quelle in cui si registrano le concentrazioni maggiori. In questo periodo, invece, in cui i livelli di traffico sono completamente ridotti, sarà interessante vedere se c'è stato qualche cambiamento ed eventualmente si potrà anche provare ad indagare sui risultati, per ricercare quali possono essere le cause dei valori registrati (riscaldamento, maggior consumo energetico, ecc..).
\\Proprio per questo calo del traffico sarebbe utile andare ad analizzare i dati di una stazione collocata vicino ad un'importante arteria stradale, confrontandoli magari con una stazione urbana, per vedere se si possono riscontrare differenze significative o se invece i valori registrati da entrambe tendono a seguire uno stesso trend.
\newpage

\section{Considerazioni generali}
Di seguito una serie di considerazioni riguardanti tutte le analisi previste.
\paragraph{Normalizzazione dei dati}
Nella descrizione delle analisi previste ho menzionato il processo di normalizzazione dei dati. Bisognerà capire, innanzitutto, se vale la pena tentare di compiere questo pre-processamento dei dati, soprattutto considerando il fatto di dover ottenere comunque dei dati rappresentativi della situazione reale, senza che vengano completamente stravolti. 
\\Se verrà scelto di utilizzare questo procedimento bisognerà poi definire come attuarlo. Sicuramente le variabili da tenere in considerazione sono diverse, ma penso che le più importanti da considerare siano: precipitazioni ed altezza del PBL. Bisognerà anche capire se sono disponibili i dati relativi a quest'ultima misura (ARPA Lombardia nel suo studio li utilizza) oppure se esiste la possibilità di stimarli basandosi su altri fattori.
\paragraph{Difficoltà nell'analisi dei risultati}
Nelle analisi previste sicuramente tra gli obbiettivi c'è quello di cercare delle relazioni tra misure adottate (blocchi del traffico, maggior utilizzo di energie rinnovabili) ed i livelli di inquinamento. Questo non sarà sicuramente un compito semplice, soprattutto per quanto riguarda quelle analisi che riguardano periodi brevi e quindi molto più soggetti alla variabilità dei livelli degli inquinanti. Il nostro lavoro dovrà quindi tenere conto di questo aspetto, cercando di allargare il più possibile l'insieme dei dati di confronto, in modo da "attutire" questa variabilità. Ad esempio, per quanto riguarda l'epidemia COVID, il confronto dei livelli registrati nel mese di Marzo dovrà essere almeno con i dati dello stesso mese degli ultimi anni e non sono con quelli dell'anno scorso, così da avere un miglior quadro della situazione.
\\Un altro aspetto che sarà importante ricordare in queste analisi è che tutti i livelli presentano dei trend e quindi bisognerà tenerne conto per avere una stima più precisa dei reali ribassamenti/annalzamenti registrati.
\\Inoltre, quando si trattano analisi che cercano di stimare l'influenza di determinate misure sulla qualità dell'aria, non sarà semplice fornire dei risultati certi e bisognerà essere cauti a non trarre conclusioni errate o influenzate da altri fattori. Sarà quindi opportuno che ogni attività prevista veda lo svolgimento di più analisi, che possano darci una certa confidenza sulla validità dei risultati ottenuti, senza cadere nel rischio di presupporre correlazioni basate solo su un semplice confronto.
\paragraph{Scelta di località simili}
Quando si dovranno fare confronti tra i valori registrati in due località e avremo necessità che queste presentino caratteristiche simili, l'idea era quella di scegliere le coppie in modo che le rilevazioni meteo per le due località presentino caratteristiche simili, ovvero almeno con temperature e livelli di precipitazione abbastanza vicino. Questa considerazione è legata al fatto che sono due fattori molto influenti sugli inquinanti e quindi cercare di avere una situazione climatica simile dovrebbe permetterci di fare dei confronti più precisi.
\newpage

\bibliographystyle{ieeetr}
\bibliography{References}
\end{document}
